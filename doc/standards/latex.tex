%------------------------------------------------------------------------------
% Syneight - A soft-realtime transaction monitor.
% Copyright (C) 2003-2004 The Syneight Group.
%
% TODO: License.
%------------------------------------------------------------------------------

\chapter{Latex}
\label{cha:latex}

Almost all \SYNEIGHT documentation is written in \LaTeX{} and then compiled
to the appropriate format like Post Script, PDF, HTML or even plain text.
For this reason we have to define some guidelines how to write the
documents to maintain consistency throughout the documents.


\section{\LaTeX{} Header}
\label{sec:latex-header}

Every \LaTeX{} file should have the following header prepended:

\begin{verbatim}
%------------------------------------------------------------------------------
% Syneight - A soft-realtime transaction monitor.
% Copyright (C) 2003-2004 The Syneight Group.
%
% TODO: License.
%------------------------------------------------------------------------------
\end{verbatim}


\section{Standard Packages}
\label{sec:standard-packages}

The standard packages that we use throughout the \SYNEIGHT documents are
the following:

\begin{verbatim}
\usepackage{fullpage}
\usepackage{varioref}
\usepackage{epic}
\usepackage{ecltree}
\usepackage{amsmath}
\usepackage{tabularx}
\usepackage{epsfig}
\usepackage{multirow}
\usepackage{hhline}
\usepackage{graphicx}
\end{verbatim}

\begin{todo}
    Describe the packages and when and where one should use them...
\end{todo}

We also have our own packages that are stored in \verb|doc/lib/syneight/|.
These have to be included in the following way:

\begin{verbatim}
\input{syneight/include.tex}
\end{verbatim}

This include file aggregates other files located in lib that contain
following things:

\begin{description}
    \item[lib/env.tex] This package contains predefined terms like
        requirement, todo, task and so on. 

    \item[lib/people.tex] This package contains commands mapping
        initials to the names of the people involved in the
        \SYNEIGHT project.

    \item[lib/abbreviations.tex] This package contains commands that
        are abbreviations of more complicated expressions like
        macro names and often used abbreviations like \MMORG and
        \NVE
\end{description}


\section{Document Structure}
\label{sec:document-structure}

All \LaTeX documents consist of one main tex file and subfiles.

The main file has the name derived from the title of the document. For
example the ``Coding Standards'' document main tex file is named
``standards.tex''.

The contents of the main file are:

\begin{description}
    \item[Document type definition] That is the first line of every
        main document file and is in most cases 
        \verb|\documentclass[11pt]{book}|

    \item[Section depth definition] That defines the depth of the
        section numbering. For example section number depth 2
        would result in numbers up to two parts: 2.5 and section
        number depth of 3 would result in three parts: 2.5.2

        In most cases one should set it to
        \verb|\setcounter{secnumdepth}{3}|

    \item[Use package definition] This is a block of definitions
        which packages to use in the document a common example
        can be found in section \ref{sec:standard-packages}

\begin{question}{what is that sloppy definition?}\end{question}

    \item[Local package definition] In this block defines which
        \SYNEIGHT project specific packages have to be included.
        This block looks always in the same way you can find it
        in section \ref{sec:standard-packages}

\begin{question}{renewcommand ???}\end{question}

\begin{question}{what is INTERFACE command definition doing here?}\end{question}

    \item[Frontsite definition] 

\end{description}


\section{Modelines}
\label{sec:latex-modelines}

Append the following \emph{modelines} (per-file configuration parameters) for
{\tt vim} and {\tt emacs} to every \LaTeX{} file:

\begin{verbatim}
%%% Local Variables:
%%% mode: latex
%%% TeX-master: "foo"
%%% tab-width: 4
%%% End:

% vim:ts=4:sw=4
\end{verbatim}

The "foo" parameter is the (double-quoted) master document a
\LaTeX{} file belongs to (without the {\tt .tex} extension).
E.g. if you have a {\tt foo.tex} file which
includes {\tt bar.tex}, then you have to use "foo" as TeX-master
in the {\tt emacs} modeline of {\tt bar.tex}.

Omit the parameter for TeX-master in the master document itself, i.e.
write 

\begin{verbatim}
%%% TeX-master:
\end{verbatim}


These modelines will ensure that

\begin{itemize}
\item the displayed TAB-width is 4 in both editors, and
\item {\tt emacs} will be put into \LaTeX{} mode
\end{itemize}


%%% Local Variables: 
%%% mode: latex
%%% TeX-master: "standards"
%%% End: 
% vim:ts=4:sw=4
