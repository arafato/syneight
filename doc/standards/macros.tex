%------------------------------------------------------------------------------
% Syneight - A soft-realtime transaction monitor.
% Copyright (C) 2003-2004 The Syneight Group.
%
% TODO: License.
%------------------------------------------------------------------------------

\chapter{Macros}
\label{cha:macros}

\section{Reasons for using Macros}
\label{sec:reasons-using-macros}

Macros are a powerful tool which is useful in C++-development with
respect to three areas:
\begin{description}
\item[Code Reflection:] The program should access some property of the
  code itself, such as file name, source line, or a string which
  represents a given C++-expression.
\item[Redundancy Reduction:] Instead of repeating a certain code
  fragment several times, macros can be used in many cases to generate
  the code properly. 
\item[Conditional Compilation:] The same source should be compiled in
  different ways. Typical usages arise in the context of platform
  portability and debugging levels. 
\end{description}
%
In all three cases, the preprocessor is one option to solve the
problem. However, there is one definite rule above all other rules
with respect to macros.
%
\begin{rule*}{Avoid Macros if possible}
  Use macros if and only if there is no other sufficient way to
  achieve the same result.
\end{rule*}
%
The key-work in the above statement is \emph{sufficient}. So what is
sufficient? We detail this issue in the following subsections.


\subsection{Code Reflection}
\label{sec:code-reflection}

First of all, code reflection does not work without macros. Thus if
you need the line number of some statement, then you need the
preprocessor. However, these situations will be covered to a vast
degree by the annotations of chapter \vref{cha:annotations}.  These
standard annotations are justified by the following rule.
%
\begin{rule*}{Use Macros for Code Reflection}
  If you need to access the file name, the base file name\footnote{The
    base file is the file which is given to a compiler to be
    compiled. In most cases, this file will be a *.cpp file, which
    includes some *.hpp files. If you write a macro in some *.hpp file,
    you can access the *.cpp file which includes this *.hpp file.\\
                                %
    For example, if you have {\tt a.hpp} using some macro {\tt
      SYNEIGHT\_BASE\_FILE\_NAME} and two files named {\tt b.cpp} and
    {\tt c.cpp}
    which include {\tt a.hpp}, then {\tt SYNEIGHT\_BASE\_FILE\_NAME} will
    expand to {\tt b.cpp} in the first case, while it expands to {\tt
      c.cpp} in the second case.},
  %
  the current source code line number, or some stringized expression.
\end{rule*}
%


\subsection{Redundancy Reduction}
\label{sec:redundancy-reduction}

Next, redundancy often occurs when you need parameter lists (ordinary
method/procedure parameters and template parameters) of variable length
-- if some type-less solution is not an option (most of the times, it
is not), then it is necessary to define the construct with every
possible parameter list, i.e., to enumerate the construct for one
parameter, two parameters etc. To avoid this redundancy, we use
macros.
%
\begin{rule*}{Use Macros for Parameter Lists of Variable Length}
  In situations where you need to define a class of a procedure which
  takes a variable number of arguments without loosing
  typing, use macros.
\end{rule*}
%
%
There might be other cases when macros are unavoidable to write code
without unwanted redundancy\footnote{While we strive for code which
  has as few redundant parts as possible, we accept redundancy if the
  language does not give us a clean way to remove it, or if the
  redundancy gives us independence which helps to maintain the overall
  architectural integrity.}. We will add the cases when they appear.


\subsection{Conditional Compilation}
\label{sec:cond-comp}

Finally, we have conditional compilation. It is hard to state a
general rule for situations which require conditional compilation. The
following rule incorporates the general indicators when conditional
compilation is necessary.
%
\begin{rule*}{Use Macros for Conditional Compilation if C++ mechanisms are
    insufficient}
  The main reasons to use conditional compilation using C++ macros are the
  following:
  \begin{itemize}
  \item C++-code would be slow.
  \item A lot of unnecessary code would be emitted.
  \item C++ cannot be used to encapsulate well the varying aspect.
  \end{itemize}
  If one of these rules matches, use conditional compilation.
\end{rule*}
%
The last case of the above rule is a bit hard to understand.
Basically, when you are varying the flow of control, e.g., introduce
new nested blocks, or add an additional exception handler, you cannot
expect to find a clean way to handle the case with C++ directly.\\
%
There is one clear cut situation when you need conditional
compilation, namely platform independence.
%
\begin{rule*}{Use Macros for Conditional Compilation to adapt to the Platform}
  To achieve platform independence with respect to a certain set of
  platforms, use conditional compilation. 
\end{rule*}

\section{General Macro Rules}
\label{sec:general-macro-rules}

There are only a few general naming convention for macros in \SYNEIGHT\ 
-- all other rules are specific to the corresponding group.
%
First, we require a prefix for all macros defined within the \SYNEIGHT
sources.
%
\begin{rule*}{Macro Prefixing}
    Every macro of \SYNEIGHT starts with {\tt SYNEIGHT}.
\end{rule*}

Macros can be conditional, such as if-then-else constructs. The
question is whether we use indentation and how. The following rules
define our indentation with respect to macros.
%
\begin{rule*}{Macro Indentation}
  After a conditional preprocessor directive (such as {\tt \#if,
    \#ifdef, \#else}\dots) the indentation has to increase or decrease
  by one step\footnote{In our case increasing/decreasing the number of
    tabs.} with respect to other preprocessor code.\\
  %
  \textbf{We indent preprocessor code and usual C++ code completely
  independently.}
  %
  Note that the {\tt \#} must be the first sign of the line, i.e., we
  use for example {\tt \# <INDENT> if}.
  %
  This rule has one exception, namely internal include
  guards\footnote{See subsection \vref{sec:macro-groups}.}. The
  directives of an internal include guard do not affect indentation.
\end{rule*}


\section{Internal Macros}
\label{sec:internal-macros}

%
If a macro is used internally, then it is prefixed with {\tt
  SYNEIGHT\_\_INTERNAL}. It is not possible to give a clear cut
definition of a macro that is internal. In principal, we say that a
macro is internal, if it used in a predefined set of source
  locations.
%
\begin{rule*}{Internal Macro Prefixing}
  Every internal macro of \SYNEIGHT starts with {\tt
    SYNEIGHT\_\_INTERNAL}.
\end{rule*}
%
An internal macros should only be used in a limited set of locations,
i.e., the programmer who writes the internal macros should \emph{know}
all direct uses of the internal macro. For example, a sub-macro of
another macro is an internal macro. A macro which is used to emit some
code in a single location is also internal. 

If the macro is used as sub-macro of some other macro, then it is used
at a limited set of locations, but it cannot be undefined: The macro
is used indirectly at yet unknown locations. 
%
On the other hand, if an internal macro does not serve as a sub-macro,
usually it can be undefined after is usage. An example for such a
situation is the following: 
%
\begin{verbatim}
#define SYNEIGHT__INTERNAL__EMIT__TO_STRING(COUNT) \
template< SYNEIGHT__BASE__EMIT__PARAMETER(typename T,COUNT) > \
::std::string to_string( SYNEIGHT__BASE__EMIT__PARAMETER2(T,COUNT,const & t)) \
{ \
    ::std::ostringstream os; \
     os SYNEIGHT__BASE__EMIT__PARAMETER_USE(<< t, SYNEIGHT__BASE__DEC(COUNT) , << ' ') \
        << t ## COUNT; \
    return os.str(); \
}
SYNEIGHT__BASE__FOR_0_TO(SYNEIGHT__INTERNAL__EMIT__TO_STRING,SYNEIGHT__BASE__FOR_0_TO_MAX);
#undef SYNEIGHT__INTERNAL__EMIT__TO_STRING
\end{verbatim}
Here, {\tt SYNEIGHT\_\_INTERNAL\_\_EMIT\_\_TO\_STRING} is used only to
emit the code and thus it should not be used afterwards. In such a
situation, we require, that the internal macro is undefined (as above)
to avoid misuse.
%
\begin{guideline*}{Limit Use of Internal Macros}
  Undef an internal macro after its usage if this is possible.
\end{guideline*}
%
There are no further rules on internal macros, they are an
implementation detail.

\section{Base Macros}
\label{sec:base-macros}

Macros which are reused in different contexts, such as helper macros
to emit a template parameter list of variable length, will be
collected into a dedicated library of base macros. Each of these
macros will be prefixed with {\tt SYNEIGHT\_\_BASE}.

\begin{rule*}{Base Macro Naming Convention}
  A base macro must be named 
  \begin{verbatim}
    SYNEIGHT__BASE__<NAME>
  \end{verbatim}
  where {\tt <NAME>} is an arbitrary string which does not contain
  more than two consecutive underscores.
\end{rule*}
%
Mostly, base macros are used to build more complex macros. In
particular, there will be a large group of macros which are used to
emit code efficiently. 


\section{Macro Groups}
\label{sec:macro-groups}

In this subsection, we subdivide macros into seven groups. This
subdivision is different from the categorization of the last section
which presented the three motivating reasons to use macros.  The
categorization presented here, is oriented towards the way a macro is
actually used.\\
%
From the point of view of a developer who uses a macro, the original
motivation for introducing the macro is not important compared to the
interface of the macro, i.e., the defined behavior of the macro. 
%
We divide the macros into seven groups:
%
\begin{description}
\item[Include Guards:] Controlling whether a file is to be included or
  not.
\item[Macros controlling conditional compilation:] Include guards are
  an example -- however we treat them as special case. So what remains
  are macros that are used as arguments in some preprocessor directive
  that controls the code to be compiled. 
\item[Macros for Code Parameterization:] These macros are used to
  parameterize a build. In contrast to a macro which controls
  conditional compilation, a parameterizing macro is not used to
  switch some feature on or off, but to set some value, e.g., the
  current debugging level.
\item[Macros used within C++-Code:] Macros which are used as ordinary
  statements within C++ code.
\item[Code Generating Macros:] Macros used to generate some C++ code
  -- the user is only using the C++ construct, i.e., the macros are
  only a way of writing this code. 
\item[Other Macros:] There remain some macros which do not belong to
  any of the categories described so far. 
\end{description}
%
Each of these groups will be governed by different rules for writing
them. Also, the user of a macro should recognize the category of a
macro by reading its name, i.e., the category will be reflected in the
name. We will present the rules of each group in an own subsection.
%


\section{Include Guards}
\label{sec:include-guards}

\begin{rule*}{Include Guard Naming Convention}
  An include guard has the following naming scheme
  \begin{verbatim}
    SYNEIGHT__DIR1__...__DIRn__FILE_HPP
  \end{verbatim}
  for an include file {\tt dir1/\dots/dirn/file.hpp}.
\end{rule*}
%
\begin{rule*}{Include Guard Position}
  The include guard is always written as 
  \begin{verbatim}
    #ifndef SYNEIGHT__DIR1__...__DIRn__FILE_HPP
    #define SYNEIGHT__DIR1__...__DIRn__FILE_HPP
    <CODE>
    #endif
  \end{verbatim}
  where {\tt <CODE>} is the code contained in the include file. 
  \textbf{Every include file has to have an include guard.}
\end{rule*}
%
This way of preventing the double inclusion of the same file into the
same compilation unit\footnote{The result of the preprocessing stage
  which is fed into the compiler} is called internal include
guarding. 
%
Another way of achieving the same logical result is called external
include guarding. In this case you write 
\begin{verbatim}
    #ifndef SYNEIGHT__DIR1__...__DIRn__FILE_HPP
    #  include <syneight/dir1/.../dirn/file.hpp>
    #endif
\end{verbatim}
The only difference is performance. In the case of internal include
guards, the preprocessor has to read the entire file, independently of
whether it is used or not. This might be unimportant in small
projects, however, in large projects, it is easy to construct
realistic cases which cause a file to included $n^2$ times if $n$
files are in the project. If this file is 40, 50 kB large, this can
slow down compilation times. 
%
Thus external include guards are used, they avoid the problem by
checking whether the file has been included before including it. 
%

The usage of include guards is limited, more precisely, include guards
are only allowed to appear in internal and external include guards. To
express the need to include some file, we will use conditional
inclusion switches, see subsection \vref{sec:cond-incl}.
\begin{rule*}{Limited use of Include Guards}
  An include guard is only allowed to appear in external and internal
  include guards.
\end{rule*}

%% In some \textbf{rare} cases, it might be necessary to place code
%% outside of the common include guard. These cases have to be defined
%% specifically and have to be used with great care. If a developer
%% thinks this is necessary, 
%
%% If you position code outside of the include guard, you need to guard
%% it explicitly with some other guard. Therefore, if code is outside of
%% an include guard, it has to be a possibly guarded include directive.

%
%% The usage of external include guards and the placement of code outside
%% of the include guards will be subject to \textbf{specific rules} described in 
%% subsection \vref{sec:relat-betw-files}.



\section{Macros Controlling Conditional Compilation}
\label{sec:macr-contr-cond}

We call a macro which controls conditional compilation a
\emph{switch.} Switches are never used directly in the code but only
for governing {\tt \#if} directives. The body (or bodies) of these
conditional statements contain further macro definitions or plain
C++-code.
%
\begin{rule*}{Conditional Compilation Switch Naming Convention}
  A macro controlling conditional compilation is called a public
  switch and is named as
  \begin{verbatim}
    SYNEIGHT__SWITCH__<NAME>
  \end{verbatim}
  where {\tt <NAME>} is an arbitrary string which matches the macro
  naming conventions and does not contain more than two consecutive
  underscores.
\end{rule*}
%
To make the term ``controlling conditional compilation'' clear, we use
the following definition.
%
\begin{definition*}{Public Switch}
  \label{def:public-switch}
  All macros which are potentially defined by some client of \SYNEIGHT
  to control the compilation of \SYNEIGHT (or one of its modules), are
  public switches (named {\tt SYNEIGHT\_\_SWITCH}).\\
\end{definition*}
%
Typically, these macros are defined as command-line arguments of the
compiler invocations to build \SYNEIGHT. Switches are global to
\SYNEIGHT, i.e., linking units which have been compiled which different
switches leads to undefined behavior.

In contrast, there are internal switches, which represent derived
decisions.
\begin{definition*}{Internal Switch}
  Internal Switches (named {\tt SYNEIGHT\_\_INTERNAL\_\_SWITCH}) are
  either derived from the public switches and/or the platform features.
\end{definition*}
%

The usage of a switch is restricted, as already sketched above. A
switch is only allowed to appear as part of a conditional preprocessor
statement and as expression in a C++-statement.
%
\begin{rule*}{Correct Usage of a Switch}
  A switch is \textbf{only} allowed to be used either as part of an
  boolean preprocessor expression which is part of a conditional
  preprocessor statement based on one of the two keywords {\tt
    \#if} and {\tt \#elif}.
\end{rule*}
%
The above rule does not allow us to use switches in {\tt \#ifdef} or
{\tt \#ifndef} statement. This is a feature, not an oversight. Code
which uses a switch should fail to compile, if the switch has no
defined value. A switch that would be based on being defined or not
being defined is always implicitly defined. The above rule makes sure,
that every use of a switch must be preceded by an explicit definition
-- even if this definition is the default definition. Such a default
definition still allows to change the default value at a central
location.\\
%
Also, we disallow the direct usage of a switch in a C++
expression. We decouple the switches from the macros which are used as
expressions, since this allows to change both interfaces
independently. For example, a switch might have the values 0, 1 and
2. These values correspond to members of an enumeration with values
$A=0$, $B=1$ and $C=2$. The enumeration is extended with a new value
$D=3$ and the 2 should be mapped to $D$ now. Such a change can be
facilitated easily, if a further level of redirection is setup from
the very beginning.  
%
\begin{rule*}{Wrap Switches into Expression- and Statement-Macros}
  \label{guide:wrap-switch}
  If you need to use the value of a switch in C++-code, then the value
  has to be wrapped into an expression/statement macro, see section
  \vref{sec:expression-macros}.
\end{rule*}


In contrast, the definition of a switch (more precisely of its value)
is allowed to use {\tt \#ifndef} and {\tt \#ifndef} statements to
provide sensible default values. These values are set in dedicated
configuration files such that all switches can be changed within a set
of well-defined files (and the next rule enforces that these files are
included in relevant compilation units).
%
\begin{rule*}{Default Values for Switches}
  Every switch must have a default value. The definition of the
  default value is always given as follows:
  \begin{verbatim}
    #ifndef SYNEIGHT__SWITCH__<NAME>
    #  define SYNEIGHT__SWITCH__<NAME> <VALUE>
    #endif
  \end{verbatim}
  Every switch has an associated definition file. This file must be
  included if the component uses the switch. 
  %
  The default value is defined directly or indirectly (by the file which
  is included) within this file.
  %
\end{rule*}
%
Following rule \vref{rule:global-config-first}, the definition file of a
switch must include a global configuration file\footnote{ {\tt
    <syneight/config/production.hpp>}, {\tt
    <syneight/config/unittest.hpp>}, or {\tt
    <syneight/config/testsupport.hpp>} -- see definition
  \vref{def:global-config-file}.}.
%
Actually, the most fundamental switches are defined in these three
files. If a switch is defined in one of these files, then is has to be
defined according to rule  \vref{rule:global-config-file-hier}:
%
\begin{rule*}{Default Values in Global Configuration Files}
  The global configuration files form a hierarchy (more precisely,
  {\tt production.hpp} is refined by {\tt unittest.hpp}, which is in
  turn refined by {\tt testsupport.hpp}).\\
  %
  If a switch is defined in a global configuration file $A$ then it
  has to be defined in all global configuration files which refine
  $A$.
\end{rule*}
%

Therefore, the global configuration files can adapt the compilation
process according to the needs of the corresponding use-case. At the same
time, not every switch has to be defined in the global configuration
files. It is important to keep the global configuration files small,
since changing one of them means recompiling and relinking vast parts
of the code -- a time-consuming task. In conclusion, a switch should
be defined in one of the global configuration files only if
necessary.
%
\begin{guideline*}{Keep Global Configuration Small: Switches}
  If a switch has the same default value in all three configuration
  cases, then do not define the switch in the global configuration
  files but in a dedicated local configuration file which include the
  corresponding global configuration file.
\end{guideline*}


In some cases, a switch must be set for a single compilation unit
only. In this case, we use a local switch.
%
\begin{rule*}{Local Switch Naming Convention}
  A macro controlling conditional compilation which is local to a
  single compilation unit is called a local switch and is named as
  \begin{verbatim}
    SYNEIGHT__LOCAL__SWITCH__<NAME>
  \end{verbatim}
  where {\tt <NAME>} is an arbitrary string which matches the macro
  naming conventions and does not contain more than two consecutive
  underscores.
\end{rule*}
%
The rules with respect to the usage of a local switch are identical to
the rules which govern the usage of public and internal
switches. Typically, local switches are set before including any
include file within the source of a corresponding compilation unit.

To summarize the differences:
\begin{itemize}
\item A public switch is compile time setting which must be set
  consistently between all compilation units linked together.
\item An internal switch is a derived setting which is not set
  directly by user but is derived by some preprocessor code. The
  client is not allowed to refer to an internal switch.
\item A local switch is a compile time setting which might be
  set differently between compilation units which are linked
  together. 
\end{itemize}

\subsection{Conditional Inclusion}
\label{sec:cond-incl}

A special case of conditional compilation is the conditional inclusion
of files. The case arises when 
\begin{itemize}
\item different compile time switches require different interfaces to
  be included,
\item it is not possible to include the corresponding files
  immediately, since some other code needs to be processed before
  including the interface.
\end{itemize}
These situations are rare -- within \SYNEIGHT we will find such
situations at the core of the configuration and annotation
framework.

\begin{rule*}{Conditional Inclusion Switch Naming Convention}
  A macro controlling the conditional inclusion of files is named as
  follows: 
  \begin{verbatim}
    SYNEIGHT__NEEDS__<NAME>
  \end{verbatim}
  where {\tt <NAME>} is the file name (complete with path). The switch
  has the value 0 if the file is not required to be included. If its
  value becomes 1, then the file has to be included. 
\end{rule*}
%
\begin{rule*}{Conditional Inclusion Switch Setting}
  A conditional inclusion switch is initialized as follows:
\begin{verbatim}
  #ifdef SYNEIGHT__NEEDS__<NAME> 
  #  error SYNEIGHT__NEEDS__<NAME> already initialized
  #else
  #  define SYNEIGHT__NEEDS__<NAME> 0
  #endif
\end{verbatim}
  If it is determined that the switch must be set to 1, the following
  code has to be used:
\begin{verbatim}
  #if SYNEIGHT__NEEDS__<NAME> == 0
  #  undef SYNEIGHT__NEEDS__<NAME>
  #  define SYNEIGHT__NEEDS__<NAME> 1
  #endif
\end{verbatim}
  Finally, a conditional inclusion switch must be applied in the
  following way. Note that it has to undefined at the end.
  TODO: Why?
\begin{verbatim}
  #if SYNEIGHT__NEEDS__<NAME> == 1
  #  include <NAME> 
  #endif
  #undef SYNEIGHT__NEEDS__<NAME>
\end{verbatim}
\end{rule*}

\begin{guideline*}{Initialization and Application of Conditional Inclusion}
  The initialization block and the application block should be located
  in the same file.
\end{guideline*}

\subsection{Optional Features}
\label{sec:optional-features}

Another special case of switches are optional features. The presence
of some features might be dependent on the platform, or on the some
other switches, e.g., a compilation switch might allow to switch
on/off a certain feature -- if it is present, some other features
might require higher overhead. Thus it has been made a compile-time
switchable.  
%
Regardless of the reason for the optional feature, we use the
following convention.

\begin{rule*}{Optional Feature Switch}
  A macro indicating (and optionally controlling) the presence of an
  optional feature is named as follows:
  \begin{verbatim}
    SYNEIGHT__HAVE__<NAME>
  \end{verbatim}
  where {\tt <NAME>} is the name of the feature. 
\end{rule*}
%
\begin{guideline*}{Optional Feature Switch Value}
  An optional feature switch should have the value 0 if the feature is
  not present and the value 1 if it is present. 
  %
  If there are more than two possible values, consider to use two or
  more switches.
\end{guideline*}




\subsection{Macros for Code Parameterization}
\label{sec:macros-for-code-parameterization}

\TODO{add stuff}

\begin{rule*}{Public Parameter}
  \label{def:public-parameter}
  A macro which is used to parameterize the compilation of some
  component of \SYNEIGHT has to adhere the following naming convention:
  \begin{verbatim} 
    SYNEIGHT__PARAM__<NAME>
  \end{verbatim}
  where {\tt <NAME>} is the name of the parameter. {\tt <NAME>} is not
  allowed to contain more than two consecutive underscores.  Such a
  macro is allowed to take parameters.
\end{rule*}


\begin{rule*}{Internal Parameter}
  A macro which is used to parameterize the compilation of some
  component of \SYNEIGHT has to adhere the following naming convention:
  \begin{verbatim} 
    SYNEIGHT__INTERNAL__PARAM__<NAME>
  \end{verbatim}
  where {\tt <NAME>} is the name of the parameter. {\tt <NAME>} is not
  allowed to contain more than two consecutive underscores.  Such a
  macro is allowed to take parameters.
\end{rule*}



\section{Macros within ordinary C++-Code}
\label{sec:macr-within-ordinary-cpp-code}

Macros that appear in regular C++-code are described in this
section. They can be further categorized into four groups.
\begin{description}
\item[Expression Macros:] Such macros should behave like procedure
  calls. 
\item[Fragments Macros:] Such macros contain a sequence of statements.
  A developer using them must be careful -- they cannot be used as
  sub-expressions or after an if-statement without enclosing braces.
\item[Block Macros:] Such macros always come as pairs, one macro to
  open the block, and another to close the block. That is to say,
  these macros setup some context.
\item[Guard Macros:] Such macros setup a guard within the surrounding
  block, i.e., they execute some code when the program counter passes
  them the first time, and execute some other code, when the
  surrounding block is left.
\end{description}


\begin{rule*}{Macros within ordinary C++-Code take arguments}
  All macro types of this section take optionally arguments. If there
  is a family of related macros which take different number of
  arguments, then these macros have to suffixed with the number of
  actual arguments.
\end{rule*}

\subsection{Expression Macros}
\label{sec:expression-macros}

The naming convention of Expression Macros is the default, i.e.,
whatever is not explicitly marked to be member of another macro group
is an expression macro. They are the most safe form of macros, thus
they should be default. 

\begin{rule*}{Expression Macro Naming Convention}
  A macro to be used as expression is named
  \begin{verbatim}
    SYNEIGHT__<NAME>
  \end{verbatim}
  where {\tt <NAME>} is an arbitrary string which does not contain
  more than two consecutive underscores is not indicating membership
  to any other macro group (e.g., {\tt <NAME>} is not allowed to start
  with {\tt SWITCH}).
\end{rule*}

These macros should behave like ordinary procedure calls to fit into
the C++-syntax as seamless as possible. 
%
\begin{rule*}{Expression Macros must behave like Procedures}
  Each macro to be used as expression must 
  \begin{itemize}
  \item require to type a ``;'' at its end,
  \item and must be a single C++-expression.
  \end{itemize}
\end{rule*}
The latter item in the above rule requires that a definition such as 
\begin{verbatim}
#define SYNEIGHT__<NAME>(A,B) A=B; B=0 
\end{verbatim}
is invalid since 
\begin{verbatim}
if(condition) SYNEIGHT__<NAME>(a,b);
\end{verbatim}
would give 
\begin{verbatim}
if(condition) a=b; b=0;
\end{verbatim}
which is certainly not intuitive. 
%
To achieve this requirement we give the following rule:
\begin{rule*}{Expression Macro Definition}
  The definition of an expression macro must either match one of the
  following forms, where {\tt <C++-Expression>} is an C++-expression
  without tailing semicolon and {\tt <C++-Statements>} is an arbitrary
  sequence of C++-statements.
  \begin{itemize}
  \item \begin{verbatim}
#define SYNEIGHT__<NAME> <C++-Expression>
\end{verbatim}
    \item 
\begin{verbatim}
#define SYNEIGHT__<NAME> do { <C++-Statements> } while(false)
\end{verbatim}
  \end{itemize}
\end{rule*}
%
If an expression macro should be a no-op, we use the following form.
\begin{rule*}{Noop Expression Macros}
  A macro to be used as an expression, which does nothing, has to be
  defined as follows:
\begin{verbatim}
#define SYNEIGHT__<NAME> ((void)0)
\end{verbatim}
 \end{rule*}


\subsection{Fragment Macros}
\label{sec:fragment-macros}

Sometimes, it is not possible to define a recurring code fragment in
terms of an expression. For example, if you need to declare a set of
interrelated variables within a block and if you need to access all of
them, you need to a sequence of statements to do so. Because of the
limited usage of such macros, we mark them as fragments.

\begin{rule*}{Fragment Macro Naming Convention}
  A fragment macro is named according to the following convention:
  \begin{verbatim}
    SYNEIGHT__FRAGMENT__<NAME>
  \end{verbatim}
  where {\tt <NAME>} is an arbitrary string which does not contain
  more than two consecutive underscores.
\end{rule*}

\begin{rule*}{Fragment Macro Definition}
  The definition of an fragment macro must match the
  following form, where {\tt <C++-Expression>} is an C++-expression
  without tailing semicolon and {\tt <C++-Statements>} is an arbitrary
  sequence of C++-statements.
\begin{verbatim}
#define SYNEIGHT__FRAGMENT__<NAME> <C++-Statements> <C++-Expression>
\end{verbatim}
\end{rule*}
%
The above definition enforces to entail an application of the macro
with a semicolon.

\begin{guideline*}{Avoid Fragment Macros}
  If possible, formulate a macro as expression macro or embed it into
  a block macro pair.
\end{guideline*}


\subsection{Block Macros}
\label{sec:block-macros}

A block macros pair has to adhere the following naming convention.
%
\begin{rule*}{Block Macro Naming Convention}
  A pair of block macros is named according to the following convention:
  \begin{verbatim}
    SYNEIGHT__BEGIN__<NAME>
    SYNEIGHT__END__<NAME>
  \end{verbatim}
  where {\tt <NAME>} is an arbitrary string which does not contain
  more than two consecutive underscores.
\end{rule*}
%
The definition of a block macro must follow the following rules:
%
\begin{rule*}{Block Macro Definition}
  The definition of a block macro pair must match the following form,
  where {\tt <C++-Expression>} is an C++-expression without tailing
  semicolon and {\tt <C++-Statements>} is an arbitrary sequence of
  C++-statements.
  \begin{verbatim}
    #define SYNEIGHT__BEGIN__<NAME> do { <C++-Statements> { <C++-Expression>
    #define SYNEIGHT__END__<NAME> } } while(false)
  \end{verbatim}
  If the {\tt <C++-Statements>} is an empty sequence, then the one
  pair of braces can be omitted. 
\end{rule*}
%
The pair is to be used in the following form, where {\tt
  <C++-Statements>} is an arbitrary sequence of C++-statements.
\begin{verbatim}
    SYNEIGHT__BEGIN__<NAME>;
    <C++-Statements>
    SYNEIGHT__END__<NAME>;
\end{verbatim}
%
The consequence of the above rule is that 
\begin{itemize}
\item each usage of a block macro pair behaves like a single
  C++-statement,
\item both, the begin and end macro, must be entailed by a semicolon.
\end{itemize}
%
Finally, sometimes, we need a noop-block definition, for example, when
some switch turned off some feature. In such a case the following rule applies. 
%
\begin{rule*}{Noop Block Macros}
  A block macro pair which does nothing, has to be
  defined as follows:
  \begin{verbatim}
    #define SYNEIGHT__BEGIN__<NAME> do { ((void)0)
    #define SYNEIGHT__END__<NAME> } while(false)
  \end{verbatim}
\end{rule*}



\subsection{Guard Macros}
\label{sec:guard-macros}

A guard macro must behave syntactically like an expression macro. Only
its naming convention and meaning is different. 
%
Guards are typically used to initialize or acquire some resource which
has to be released at the end of the surrounding block. Depending on
the concrete context, we will use plain C++-guards or guards which are
wrapped into macros. The latter case occurs, when we need
compile-time switching. 
%
\begin{rule*}{Block Macro Naming Convention}
  A guard macro is named according to the following convention:
  \begin{verbatim}
    SYNEIGHT__GUARD__<NAME>
  \end{verbatim}
  where {\tt <NAME>} is an arbitrary string which does not contain
  more than two consecutive underscores.
\end{rule*}

\begin{remark*}{Guards typically come with block macro versions}
  It is frequent pattern, to have a block macro pair and a guard which
  implements the same functionality. Let us assume, we have a guard 
  \begin{verbatim}
    SYNEIGHT__GUARD__<NAME>
  \end{verbatim}
  then the following block pair might be a useful additional feature:
  \begin{verbatim}
    #define SYNEIGHT__BEGIN__<NAME> do { SYNEIGHT__GUARD__<NAME>; { ((void)0)
    #define SYNEIGHT__END__<NAME> } } while(false)
  \end{verbatim}
  If the block variant is not defined, the user might be forced to
  write frequently 
  \begin{verbatim}
    { 
      SYNEIGHT__GUARD__<NAME>; 
      <C++-Statements>
    }
  \end{verbatim}
\end{remark*}


\section{Code Generating Macros}
\label{sec:code-gener-macr}

The macros which are used to generate code can be divided into two
groups: The macros which are specifically programmed to generate some code,
and the macros which are used in general for code generation. 

The first group is always used locally, i.e., the have to be undefined
after their usage. There are no other general rules governing these
macros, besides their naming.

\begin{rule*}{Local Code Generating Macro Naming Convention}
  A locally used code generating macro is named according to the
  following convention:
  \begin{verbatim}
    SYNEIGHT__INTERNAL__EMIT__<NAME>
  \end{verbatim}
  where {\tt <NAME>} is an arbitrary string which does not contain
  more than two consecutive underscores.
\end{rule*}

\begin{rule*}{Undefine Code Generating Macros}
  After using a locally used code generating macro, undefine it.
\end{rule*}

Second, there are the code generation macros which are reused. For
example, a macro which can be used to generate a template parameter
list will be useful in more than one context. Such macros are
collected are integrated in the library of base macros. 
%
\begin{rule*}{Reusable Code Generating Macro Naming Convention}
  A reusable code generating macro is named according to the
  following convention:
  \begin{verbatim}
    SYNEIGHT__BASE__EMIT__<NAME>
  \end{verbatim}
  where {\tt <NAME>} is an arbitrary string which does not contain
  more than two consecutive underscores.
\end{rule*}



\section{Namespace Macros}
\label{sec:namespace-macros}

We use the following macros to begin and end namespaces. The reason
for using macros is twofold.

\begin{itemize}
\item First, it does not allow to close a namespace without writing its
  name (the closing \} is not very verbose),
\item and second, most editors make a fuss with respect to
  indentation if {\tt namespace <NAME> \{} is written into the source
  explicitly.
\end{itemize}

\begin{rule*}{Namespace Begin and End}
  To place a code fragment {\tt <CODE>} inside of a namespace {\tt
  <NAME>} we use the following construct:
  \begin{verbatim}
    SYNEIGHT__BEGIN__NAMESPACE__<NAME>;
    <CODE>
    SYNEIGHT__END__NAMESPACE__<NAME>;
  \end{verbatim}
  The macros are defined as follows:
  \begin{verbatim}
    #define SYNEIGHT__BEGIN__NAMESPACE__<NAME> namespace <NAME> {
    #define SYNEIGHT__END__NAMESPACE__<NAME> }
  \end{verbatim}
\end{rule*}



\section{Summary}
\label{sec:summary}

The following table summarizes all macro types which we will use in
\SYNEIGHT. The fourth column, labeled \textbf{Arg}, describes
whether a macro of the corresponding type is allowed to take
arguments.

\begin{center}\begin{footnotesize}\begin{tabular}{||l|p{5cm}|p{7cm}|c||}\hhline{|t:====:t|}
\multicolumn{4}{||c||}{\textbf{Macro Types}}\\ \hhline{||----||}
\textbf{Type}         & \textbf{Naming}& \textbf{Usage} &
\textbf{Arg}\\ \hhline{|:====:|}
Include Guard & 
{\tt SYNEIGHT\_\_<NAME>\_\_INCLUDE\_GUARD }
&    
to ensure that a file is included at most once.
& no \\ \hhline{||----||}

Public Switch & 
{\tt SYNEIGHT\_\_SWITCH\_\_<NAME>}
&    
the public compile-time switches. Must be set consistently between all
compilation units to be linked together.
& no \\ \hhline{||----||}

Internal Switch & 
{\tt SYNEIGHT\_\_INTERNAL\_\_SWITCH\_\_<NAME>}
&    
to control conditional compilation based on platform features and/or public switches.
& no \\ \hhline{||----||}

Local Switch & 
{\tt SYNEIGHT\_\_LOCAL\_\_SWITCH\_\_<NAME>}
&    
to control conditional compilation for a single compilation
unit. Typically set within the source of the corresponding unit.
& no \\ \hhline{||----||}


Conditional Inclusion & 
{\tt SYNEIGHT\_\_NEEDS\_\_<NAME>}
&    
used to control the inclusion of files which cannot be included immediately.
& no \\ \hhline{||----||}

Optional Feature & 
{\tt SYNEIGHT\_\_HAVE\_\_<NAME>}
&    
to represent features which are optionally present. The features can
be enabled/disabled based on compile time switches and/or platform features.
& no \\ \hhline{||----||}

Public Parameter & 
{\tt SYNEIGHT\_\_PARAM\_\_<NAME>}
&    
the public compile-time parameters. Must be set consistently between
all compilation units which are linked together.
& yes \\ \hhline{||----||}

Internal Parameter & 
{\tt SYNEIGHT\_\_INTERNAL\_\_PARAM\_\_<NAME>}
&    
compile time parameters which are derived from public parameters and/or
the platform settings -- or parameters which are exist only for adaptability.
& yes \\ \hhline{||----||}


Expression & 
{\tt SYNEIGHT\_\_<NAME>}
&    
to be used inside of ordinary C++-code; guaranteed to behave like a
procedure call.
& yes \\ \hhline{||----||}

Fragment & 
{\tt SYNEIGHT\_\_FRAGMENT\_\_<NAME>}
&    
to be used inside of ordinary C++-code; contains a sequence of C++-statements.
& yes \\ \hhline{||----||}

Block & 
{\tt SYNEIGHT\_\_BEGIN\_\_<NAME>}
{\tt SYNEIGHT\_\_END\_\_<NAME>}
&    
to be used to open/close a block which has some special properties
defined by the macro pair.
& yes \\ \hhline{||----||}

Guard & 
{\tt SYNEIGHT\_\_GUARD\_\_<NAME>}
&    
to setup a guard within the current block.
& yes \\ \hhline{||----||}

Reusable Code Generation & 
{\tt SYNEIGHT\_\_BASE\_\_EMIT\_\_<NAME>}
&    
to emit code for the sake of redundancy elimination -- reusable macro.
& yes \\ \hhline{||----||}

Local Code Generation & 
{\tt SYNEIGHT\_\_INTERNAL\_\_EMIT\_\_<NAME>}
&    
to emit code for the sake of redundancy elimination -- used locally.
& yes \\ \hhline{||----||}


%% Base is like internal -- its not a group of macros...
%%
%%Base & 
%%{\tt SYNEIGHT\_\_BASE\_\_<NAME>}
%%&    
%%to be used by other macros as basis; fundamental preprocessor primitives.
%%& yes \\ \hhline{||----||}

Namespace  & 
{\tt SYNEIGHT\_\_BEGIN\_\_NAMESPACE\_\_<NAME>}
{\tt SYNEIGHT\_\_END\_\_NAMESPACE\_\_<NAME>}
&    
to open/close a namespace scope.
& no \\ 
\hhline{|b:====:b|}
\end{tabular}\end{footnotesize}\end{center}



%%% Local Variables: 
%%% mode: latex
%%% TeX-master: "standards"
%%% End: 
% vim:ts=4:sw=4
