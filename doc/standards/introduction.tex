%------------------------------------------------------------------------------
% Syneight - A soft-realtime transaction monitor.
% Copyright (C) 2003-2004 The Syneight Group.
%
% TODO: License.
%------------------------------------------------------------------------------

\chapter{Introduction}
\label{cha:introduction}


\section{The Goal}
\label{sec:goal}

\SYNEIGHT will be implemented in C++. C++ is not the best language
around, but it is the appropriate vehicle for the task at hand for
several reasons, most notably, it allows the implementation to exploit
low-level mechanisms, and it allows to provide high-level
abstractions.  The problems inherent to C++ projects arise for the
very same reasons -- developers can place their implementation at many
different abstraction levels.  In most cases, there is a whole variety
of design and implementation strategies to choose from.

%
This document will establish the coding guidelines to be adhered
within the \SYNEIGHT project. 
%
Coding guidelines cannot define how a component\footnote{See
  Definition \vref{def:component} for the meaning of ``component'' in
  the context of \SYNEIGHT.} (or a group of cooperating components)
must be designed, but they do define \emph{formal criteria} a
component must fulfill.
% 
For example, we will define the physical organization of a component
(i.e., which files constitute a component), or how we state
assertions.



\section{Software as Diagram}
\label{sec:software-as-diagram}

Specifications, architecture and design are established to guide the
development at large, i.e., they define \emph{components}, their
\emph{externally visible behavior}, and their \emph{interrelations
  with respect to several structures}. Examples for such structures
are module decomposition, module layering, shared objects, allocation
of threads to tasks and their communication protocols, allocation of
processes to tasks and their communication protocols, and allocation
of developers to components and modules.\\
%
The concept of the externally visible behavior is often much more
involved than it appears at first. What is externally visible? If the
component is modeled as a class, it might be the signature of their
publicly accessible methods. This signature is certainly part of the
externally visible behavior -- but it is \emph{only part} of the
behavior. The memory consumption, the thread awareness, the exception
safety, the execution time, the pre/post conditions, the internally
used assertions, and the log messages produced during execution of a
method are certainly also part of the behavior which is relevant for
the developer of a component who uses the component in concern.
Depending on the nature of the usage pattern, different aspects are
more or less important to the user. Thus, it is hard to define the set
of characteristics which are relevant in the \emph{interface
  definition -- in the sense we will use this term henceforth:}
\begin{definition*}{Interface}
  The interface of a component is the externally visible and
  explicitly defined behavior of this component.
\end{definition*}
%
We discussed the term externally visible, however, the definition
above is also requiring that the properties of an interface are
explicitly defined. Therefore, every property which is not explicitly
defined as a property of the interface is not a property of the
interface, i.e., it can change at any point in time.\\
%
Certainly, it is not possible to define the complete external
observable behavior for every component. And there are a lot of common
properties -- since we want to have a feeling of uniformity within our
components. 
%
Thus, standards are necessary to define the basic properties of
components.

But what is a component? A component is the smallest unit which is
relevant in terms of software architecture. This is somewhat
arbitrary -- we will use the following definition.
\begin{definition*}{Component}
  A component is a cohesive implementation unit which is small enough
  such that it can be rewritten easily from scratch.
\end{definition*}
%
Basically, the definition aims at the following goal: If a component
does not work properly, it can be discarded and rewritten easily. Being
able to discard a module means that it is also possible to reuse the
surrounding modules independently.\\
%
It is not necessary to cope with some old buggy code -- it is possible
to discard a component if necessary. What easily means, depends on the
context of the project -- we will use one week of work by one
developer as definition of easy.\\
%
%
We used the term \emph{relevant in terms of software architecture}
beforehand. Thus, the goal of software architecture is to provide a
blueprint for the components which will make up the final software
system -- or in our case, the initial software system. The components
of a system are not interrelated by one structure, but a whole set of
structures. This gives the definition of software architecture, we
will use in this project.
%
\begin{definition*}{Software Architecture}
  Software architecture describes the components of a software system,
  their externally visible behavior, and their interrelationships with
  respect to a set of structures.  
\end{definition*}
%
So, it is our goal to come up with a software architecture which
allows us to develop the components mostly independent of each other.
And we expect that they do work correctly together at the end, if they
adhered the rules which were established by the software
architecture.

%% \CS: To be expanded when annotations are done
%%
%%
%% Let's leave the introduction for the time being
%%
%% One more word about software in the large: Besides classical
%% functional requirements, one also needs to state \emph{qualities} of
%% software, i.e., properties which cannot be localized within a
%% restricted set of components. A quality is a property that affects the
%% software system as a whole. 
%% %


\section{Software as Code}
\label{sec:software-as-code}

%
It is infeasible to constrain every aspect of the interface of a
component explicitly. Especially, every developer has a set of basic
assumptions. A class which does not obey these assumptions is
cumbersome to use, since these assumptions are the least to be checked
for violation. Especially C++ gives the developer the possibility to
tamper with these basic rules of conduct.
%

Thus this document will make these assumptions explicit and establish
the conditions to be met such that these assumptions are met by an
actual implementation. The coding guidelines are aimed at the
following three goals:
\begin{itemize}
\item Ensure that components behave in an intuitive way and are
  reusable in broad context.
\item Ensure that components are true abstractions which are easily
  accessible, i.e., a developer using a new component should be able
  to do so quickly. 
\item Ensure that architectural constraints on a component can be
  validated effectively.
\end{itemize}
%


\section{Organization}
\label{sec:organization}


We can group the coding guidelines according to different aspects.
Some deal with intrinsic aspects of the C++ language, such as copying
semantics and integer propagation, other deal with testability and
audibility. Documentation is governed by standards, too. Finally,
there are some guidelines which are not directly reflected in the
code, but in the way, we develop our code.

\begin{description}
\item[Component Structure:] Each component will be established by a
  set of files, namely by the interface (*.hpp), the implementation
  (*.cpp / *.impl.hpp) and the test (*.t.cpp) file. 
\item[C++ Language:] Here we will rely mostly on
  \emph{Large Scale C++ Software Design} by John Lakos --
  but we will add our own set of guidelines within this document. 
\item[Annotations:] We will use a rich set of code annotations. They
  will partly replace implementation documentation -- instead of
  writing that some condition should be met, we implement a test that
  checks this condition. Based on three debug levels, we will be able
  to implement even very costly checks.\\
  %
  The very same annotations will allow us to trace program executions,
  analyze their interactions, test the correctness of components, and
  benchmark our implementations.
\item[Documentation:] We document the implementation with {\tt doxygen}, a
  tool for automated documentation generation, based on structured
  comments.
\item[Latex Standards:] We will use the same structure for all (at
  least for all bigger) documents. To make editing easy, we will set
  forth a set of guidelines.
\item[General Development Rules:] One example for such a rule: ``There
  is no intermediate -- code is designated to be either reused or
  rewritten. Make that explicit in any component.''
\end{description}


We will use the following terms within this document:
%
\begin{description}
\item[Rule:] A rule has to be obeyed in all cases. There are \emph{no}
  exceptions.
\item[Guideline:] A guideline is a rule which should be adhered whenever
  possible. 
\item[Definition:] Self-explaining. In this document, we do not use
  definition too intensively, a lot of times, we will formulate rules
  which define terms implicitly. 
\end{description}

\begin{rule*}{Breaking Guidelines}
  If a developer thinks that it is unavoidable to break a guideline, a
  brief argumentation for the need to do so must be added to the code.
  Also, a warning must be added to the documentation if the violation
  of guideline affects parts of the corresponding component interface.
\end{rule*}

%%% Local Variables: 
%%% mode: latex
%%% TeX-master: "standards"
%%% End: 
% vim:ts=4:sw=4
