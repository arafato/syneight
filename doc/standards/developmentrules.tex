%------------------------------------------------------------------------------
% Syneight - A soft-realtime transaction monitor.
% Copyright (C) 2003-2004 The Syneight Group.
%
% TODO: License.
%------------------------------------------------------------------------------

\chapter{Development Rules}
\label{cha:development-rules}

The Development Rules for \SYNEIGHT are assimilated from the book
\emph{Large Scale C++ Software Design} by John Lakos.
We will present these rules without any explanation. If any rule
needed further information for the reader, one can read the previously
mentioned book.

\section{Accessibility}
\label{sec:accessibility}

\begin{guideline*}{Data Members should be Private}
  Keep class data members private.
\end{guideline*}
%
So when is it allowed to keep class data members protected or public?
When you define a class which is only intended for component internal
use.  Such a class must be in an internal namespace. Mostly, such
classes will appear in cpp files or in impl.hpp files of complex
template classes. Rephrased, we get the following rule:
%
\begin{rule*}{Classes Public Data Members} 
  If a class has non-private data members, it must be private to a
  component.
\end{rule*}


\section{Where to Place What}
\label{sec:where-place-what}

\begin{rule*}{External Linkage: Data}
  Avoid data with external linkage at file scope.
\end{rule*}
%
To clarify, this means, that you should not declare any global
variable. If you feel the need for some global data, encapsulate into
a singleton or a facility. 

\begin{rule*}{Free Functions: External Linkage}
  Avoid free functions (except operator functions) at file scope in .hpp
  files; avoid free functions with external linkage (including operator
  functions) in .cpp files.
\end{rule*}
%
What is a free function? A function not part of a class. File Scope is
the global scope, i.e., the space outside any namespace. Inside a
namespace, you are allowed to define a free function, \textbf{but} you
must use an appropriate entry in a corresponding header file. 
%

To enforce internal linkage, you can place entities in the
\textbf{unnamed} namespace, i.e., {\tt namespace \{ ... \}}. 

\begin{rule*}{Unnamed Namespace in cpp-files only}
  The unnamed namespace is only allowed to appear in cpp-files. 
\end{rule*}

\begin{rule*}{}
  Avoid enumerations, typedefs, and constants at file scope in .hpp
  files -- consequently, if you need them in a hpp file, place them
  either in a namespace, or \textbf{much} better place them within a
  class.
\end{rule*}


\begin{rule*}{Definition \& Declaration: Where to place}
  Only classes (also templated), structures, unions, and free operator
  functions (also templated) should be \textbf{declared} at file scope
  in a .hpp file; only classes, structures, and unions should be
  \textbf{defined} at file scope in a .hpp file. 
  
  Non-inline Template functions (member functions of template classes)
  should be \textbf{defined} in a .impl.hpp file.
  %
  Inlined functions (template or not) should be \textbf{defined} in a
  .hpp file.
  
  See rule \vref{rule:files-component} for details on the files of a
  component.
\end{rule*}
%
A definition defines an entity, a declaration is only stating that
such and such an entity exists. For example, a function declaration
{\tt int func()} declares a function named {\tt func} taking no
arguments and returning an {\tt int}.\footnote{Plain C allowed to
  declare functions without any signature, i.e., without specifying
  the return value and the arguments the function will take. An
  invitation for long debugging sessions.} A corresponding definition
would be {\tt int func() \{ return 42; \}}. For details, check out 
``Large-Scale C++ Software Design''. 
% LAKOS

\begin{rule*}{Include Guards}
  Place a unique and predictable (internal) include guard around the
  contents of each header file. See section \vref{sec:include-guards}
  for details.
\end{rule*}

\begin{rule*}{Accessing Externally Linked Entities}
  Avoid accessing a definition with external linkage in another
  component via a local declaration; instead, include the .hpp file for
  that component.
\end{rule*}
%
Rephrased, this rule says, that you should not emulate the .hpp by
just writing the declarations you need directly. 


\section{Dependencies}
\label{sec:dependencies}

\begin{rule*}{Cyclic Dependences}
  Avoid cyclic dependencies among packages, components or other source
  units (such as libraries).
\end{rule*}
%
Nothing is worse than cyclic dependencies. Cycles make software
unmaintainable. \textbf{Never!} If you have a cycle and you do not
know how to resolve the cycle, tell me.  

\begin{rule*}{}
  Avoid cyclic physical dependencies among components.
\end{rule*}


\section{Special Rights of {\tt main}-Component}
\label{sec:special-rights-tt}

\begin{rule*}{New \& Delete}
  Only the .cpp file that defines \texttt{main} is authorized to
  redefine global new and delete.
\end{rule*}
%
We define debugging new \& delete operators in special include files
which are included through {\tt syneight/base/config/unittest.hpp}.
This file is only allowed to be included in a .t.cpp file, i.e., in a
file that actually contains a {\tt main}.


\section{Clean Shutdown}
\label{sec:clean-shutdown}

\begin{guideline*}{Free Memory allocated by Static Entities}
  Provide a mechanism for freeing any dynamic memory allocated to static
  constructs within a component.
\end{guideline*}
%
Whenever possible. The problem associated with not freeing static
memory is that you cannot use tools like purify to check the memory
behavior of the application. We will use our own framework for memory
testing, thus we can deal with some of the problems associated with
not freeing statically maintained memory.  Still, for the sake of
clarity and cleanness, you should free your statically maintained
memory. One way to do so is the {\tt ::std::atexit} construct (check
it out, if you do not know what it is). 
%
An example of static entity that might retain its allocated memory is
a memory management: If it free its memory, some other static entity
might produce a segmentation fault. 

We will have build our own {\tt Lifetime\_Management} (and a {\tt
  Base\_Lifetime\_Management} for {\tt syneight/base}) class which
takes care of shutting down our static entities. 


\section{Naming Convention}
\label{sec:naming-convention}


\begin{rule*}{Marking Member Files, Protected \& Private Fields}
  Use a \texttt{m\_} prefix to highlight member fields. Use a
  \texttt{p\_} prefix to highlight protected methods and a \texttt{P\_}
  prefix to highlight private methods.
\end{rule*}

\begin{rule*}{Class Names}
  Use uppercase words separated by underscores for class names (i.e.
  Class\_XYZ).
  Use an additional \texttt{\_t} postfix to distinguish type names (i.e.
  Type\_XYZ\_t).
\end{rule*}

\begin{rule*}{Enums, Constants \& Preprocessor}
  Use all caps to identify immutable values such as enumerators, const
  data and preprocessor constants (see chapter \vref{cha:macros} for
  our macro rules).
\end{rule*}

\begin{guideline*}{}
  Functions that answer a yes-or-no question should be named in an
  appropriate way (i.e. {\tt is\_valid}, {\tt has\_value}). Also a
  public {\tt int abstract\_state} can be used to define a bit
  pattern.
\end{guideline*}


\section{Arguments}
\label{sec:arguments}

\begin{rule*}{Passing User Defined Types}
  Never pass a user-defined type (i.e. class, struct, or union) to a
  function by value.
\end{rule*}

%% \begin{guideline*}{Passing PODs to Non-Inlined Functions}
%%   Avoid declaring parameters passed by value to a function as const --
%%   if the function is non-inline.
%% \end{guideline*}

\begin{guideline*}{Passing PODs}
  Plain old data (POD) such as {\tt int} should be passed as {\tt
  const} if possible. 
\end{guideline*}

\begin{guideline*}{}
  Avoid storing the address of any argument to a function in a location
  that will persist after the function terminates; pass the address of
  the arguments instead.
\end{guideline*}

\begin{guideline*}{}
  Whenever a parameter passes its argument by reference or pointer to a
  function in a location that neither modifies that argument nor stores
  its writable address, that parameter should be declared const.
\end{guideline*}

\begin{guideline*}{}
  Be consistent about returning values through arguments (e.g., avoid
  declaring non-const reference parameters).
\end{guideline*}

\section{Casting}
\label{sec:casting}

The old story -- avoid casts:
\begin{guideline*}{}
  Consider avoiding "cast" operators, especially to fundamental integral
  types; instead make the conversion explicit.
\end{guideline*}


\begin{rule*}{C-Cast}
  \textbf{Never!} Having said this, a C-cast is statement such as {\tt
    (Cat*) mouse}. Use the C++-casts ({\tt static\_cast}, {\tt
    dynamic\_cast}, {\tt const\_cast}, and {\tt
    reinterpret\_cast}). 
\end{rule*}
%
If you do not know them, read their definition. A common
misunderstanding is that the C-cast and {\tt static\_cast} are the
same. In fact, they are used in comparable situations, but they
\textbf{are not the same}. Classes need (sometimes) special treatment,
and {\tt static\_cast} provides this special treatment.

If you do not understand {\tt reinterpret\_cast} after reading about
it, do not care too much -- since
\begin{rule*}{{\tt reinterpret\_cast}}
  Do not use it! If you have to cast from type {\tt X*} to type {\tt
    Y*} and these types are not related to each other in any way (thus
  {\tt static\_cast} and {\tt dynamic\_cast} are not usable directly),
  use {\tt static\_cast<X*>(static\_cast<void*>(y))} where {\tt y} is
  of type {\tt Y*}.
  
  And -- \textbf{avoid the above construct}. It is only to be used in
  well defined situations where you want to hide a type. 
\end{rule*}

\begin{rule*}{}
  Think twice before casting away {\tt const} (using {\tt const\_cast}).
\end{rule*}
%
Casting away a {\tt const} might cause long debugging hours -- since
\begin{itemize}
\item you broke the contract of some interface, and some other
  component might become angry about this disobedience,
\item you might cause a situation where the compiler is allowed to
  behave arbitrarily -- have fun finding the bug. Even worse, such a
  situation might arise a long time after the casting code has been
  written, when some other change is done.
\end{itemize}

\section{Return Values}
\label{sec:return-values}

\begin{rule*}{Error Code}
  For functions that return an error status, an integral value of 0
  should always mean success.
\end{rule*}

\begin{rule*}{{\tt const} Return}
  Avoid declaring results returned by value from functions as {\tt
    const} (to clarify: this rule does not apply to {\tt const \&}).
\end{rule*}


\section{Default Values}
\label{sec:default-values}

\begin{rule*}{Default Values and {\tt virtual} functions}
  Avoid default arguments that require the construction of an unnamed
  temporary object in virtual functions.
\end{rule*}
%
The above rule can be simplified to:
%
\begin{rule*}{Do not use default values}
  Do not use default values. 
\end{rule*}
%
Default values such as {\tt int foo(int const bla=5)} appear to be elegant. But
they come with a lot of problems if you leave the level of small scale
projects: Having the line {\tt int foo(int const bla=5)} in an interface means
that 
\begin{itemize}
\item 5 is the default until you change the interface. If 5 is just
  some suitable value (e.g., for an initial buffer size), this might
  not be too drastic. But if 5 causes some special behavior (think
  about {\tt NULL} default arguments), then committed to the constant
  5 as a special value.
\item You cannot choose another implementation for the signature {\tt
    int foo()}. This case is more serious. Often, during development
    you find out that a default value might be helpful, to avoid
    writing too long source lines. Then you want to optimize the case,
    and you have to break the interface.
\end{itemize}
%
So what to do? Use {\tt int foo(int const bla); int foo()}. In the
implementation of {\tt int foo()}, you can still call {\tt int
  foo(5)}. If that is too slow, a specific implementation might be
necessary anyhow.

A final remark: If you use some function defined as {\tt int foo(int
  const bla=5)} and call it with {\tt foo}, then the 5 is substituted
at the place \emph{where you call} the function. The called function
knows nothing about being called with a default argument (thus if the
function definition is identical to its declaration, the definition is
not allowed to contain a default value).


\section{Constructors, Destructors, \& Assignments}
\label{sec:constr-destr-}

\begin{rule*}{Constructor \& Assignment}
  Explicitly declare (either public or private) the constructor and
  assignment operator for any class defined in a header file, even when
  the default implementations are adequate. Only exception: A class
  without fields.
\end{rule*}

\begin{guideline*}{Explicit Copy Constructor}
  If you define a copy constructor (a constructor which takes one
  argument), then use the {\tt explicit} keyword if you want to avoid
  an implicit conversion. 
\end{guideline*}
%
If you do not understand the above rule, check out the {\tt explicit} keyword.

\begin{rule*}{Virtual Destructor}
  In every class that declares or is derived from a class that declares
  a virtual function, explicitly declare the destructor as the first
  virtual function in the class and define it out of line.
\end{rule*}

\begin{rule*}{Non-Virtual Destructor}
  In classes that do not otherwise declare virtual functions, explicitly
  declare the destructor as non-virtual and define it appropriately
  (either inline or out-of-line). 
\end{rule*}

\begin{guideline*}{}
  �void depending on the order in which data members are defined in an
  object during initialization.
\end{guideline*}



\section{Friendship}
\label{sec:friendship}


\begin{rule*}{}
  Avoid granting (long-distance, i.e., to some entity defined in some
  other component) friendship to a logical entity defined in another
  component.
\end{rule*}

\begin{guideline*}{}
  Avoid granting friendship to individual functions.
\end{guideline*}

\section{Document, be Explicit, be Consistent}
\label{sec:document-be-explicit}

\begin{rule*}{}
  Document the interfaces so that they are usable by others; have at
  least one other developer review each interface.
\end{rule*}


\begin{rule*}{}
  Explicitly state conditions under which behavior is undefined.
\end{rule*}

Some other rules, which are not in the scope of the book, should be
mentioned:

\begin{rule*}{Throw declarations}
  Never use the throw statement in declarations or definitions. Rather
  write a comment naming the expected exceptions. If a external
  parent class should define an empty throw() (mostly the case in
  standard exceptions), use the Macro
  \texttt{SYNEIGHT\_\_SWITCH\_\_EXCEPTION\_NO\_THROW} to implement it.
\end{rule*}


\begin{rule*}{}
  Be consistent.
\end{rule*}

\begin{rule*}{}
  Be consistent about names used in the same way. This will develop
  during the expansion of the project.
\end{rule*}


\section{{\tt const}}
\label{sec:tt-const}

\begin{guideline*}{Use {\tt const}}
  Whenever it makes sense, use {\tt const}. It helps the compiler to
  optimize and it helps the developer to understand the code -- and it
  helps to avoid errors. 
\end{guideline*}

\begin{guideline*}{}
  Every object in a system should be const-correct. A system should be
  const-correct.
\end{guideline*}
%
That means that the following rule is satisfied by every object and
thus by the whole system.

\begin{rule*}{}
  The result of calling a const member function should not alter any
  programmatically accessible value in the object.
\end{rule*}
%
If you read this, and you do not know, what it means, then check out
the keyword {\tt mutable}. 




\section{Misc}
\label{sec:misc}


\begin{rule*}{}
  Prefer object-specific over class-specific memory management.
\end{rule*}
%
In other words, avoid overwriting {\tt new} and {\tt delete} of a
class. 

\begin{guideline*}{}
  Clients should include header files providing required type definitions
  directly; except for non-private inheritance, avoid relying on one
  header file to include another.
\end{guideline*}

\begin{rule*}{}
  Never attempt to delete an object passed by reference. Never use
  pointer arithmetic to a reference.
\end{rule*}

\begin{rule*}{}
  When implementing memory management for a general, parameterized
  container template, be careful \textbf{not} to use the assignment
  operator of the contained type when the target of the assignment is
  uninitialized memory.
\end{rule*}

\begin{guideline*}{}
  When implementing memory management for a completely
  general, parameterized container that manages the memory for its
  contained objects, assume that the parameterized type defines a copy
  constructor and a destructor - nothing more.
\end{guideline*}


\begin{rule*}{}
  Avoid hiding a base-class function in a derived class.
\end{rule*}

\begin{guideline*}{}
  The semantics of an overloaded operator should be natural, obvious,
  and intuitive to clients. The syntactic properties of overloaded
  operators for user-defined types should mirror the properties already
  defined for the fundamental types.
\end{guideline*}

\begin{guideline*}{}
  Avoid using short, unsigned or long in the interface; use int instead.
\end{guideline*}

\begin{guideline*}{}
  Use double exclusively for floating-point operations. Use float only
  for performance reasons.
\end{guideline*}

\begin{guideline*}{}
  Where possible, implement templates on top of a factored reusable
  void * pointer type using inline functions to reestablish type safety.
\end{guideline*}

\begin{guideline*}{}
  Avoid allowing programmatic access to physical values.
\end{guideline*}

\begin{guideline*}{}
  Consider not using unsigned even in the implementation.
\end{guideline*}

\begin{guideline*}{}
  Avoid using preprocessor macros in header files except as include
  guards.
\end{guideline*}

%% \begin{guideline*}{}
%%   Clients should include header fies providing required type definitions
%%   directly; except for non-private inheritance, avoid relying on one
%%   header file to include another.
%% \end{guideline*}

\begin{guideline*}{}
  A component x should include \texttt{y.hpp} only if x makes direct
  substantive use of a class or free operator function defined in y.
  Exception: use-case includes.
\end{guideline*}

\begin{guideline*}{}
  In general, avoid granting one component license that, if also taken
  by other components, would adversely impact the system as a whole.
\end{guideline*}

\begin{guideline*}{}
  Prefer modules to non-local static instances of objects, especially
  when:\begin{itemize}
    \item {Direct access to the construct is needed outside a
        translation unit.}
        \item{The construct may not be needed during start-up or immediately
        thereafter and the time to initialize the construct itself is
        significant.}
  \end{itemize}
\end{guideline*}



%%% Local Variables: 
%%% mode: latex
%%% TeX-master: "standards"
%%% End:
% vim:ts=4:sw=4
