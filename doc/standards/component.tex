%------------------------------------------------------------------------------
% Syneight - A soft-realtime transaction monitor.
% Copyright (C) 2003-2004 The Syneight Group.
%
% TODO: License.
%------------------------------------------------------------------------------

\chapter{Components}
\label{cha:components}


\TODO{The notion of an Interface -- a component can contain classes,
  references, macros etc. Note that somewhere. These items can be part
  of the interface or declared to be internal. Note that too.}

\section{What is a Component?}
\label{sec:what-is-a-component}

We will try to hold the content of a component in a very tight context.
Therefore we regard one class per component as optimum.
%
\begin{definition*}{Component}
  \label{def:component}
  A component is code, atomistic in reuse, files and interface. This means,
  you should not be able to reasonable split the code without losing those
  values.
\end{definition*}

\begin{definition*}{Contents of a Component}
  \label{def:contents-component}
  A Component contains any type of coherent and cohesive code, such as classes,
  references, macros, etc. This can also imply the use of more than one type.
\end{definition*}
%
The main task of a component is to section the code into contextual parts which
can be separated easily from the \SYNEIGHT code.
%
\begin{rule*}{Components are reusable}
  \label{rule:components-portable}
  A Component is independent enough to be separated from the higher context it
  stands in and to be used anywhere, even in another project. This
  reusability may not extend to requirements or dependencies such as virtual
  classes.
\end{rule*}

\section{Files of a Component}
\label{sec:files-component}

We will use three files in a Component (subsequently called
\texttt{component}), a C++ Header file, a C++ implementation file and a test
suite for the component.
%
A fourth file may prove necessary due to the fact that C++ templates might get
compiled each time they are \texttt{\#include}d, whether they are used or not.

\begin{rule*}{Files of a Component}
  \label{rule:files-component}
  A component consists generally of 3 files, a 4th file is mandatory if
  templates are used.
  
  \begin{itemize}
    \item {\texttt{component.hpp} - Interface}
    \item {\texttt{component.impl.hpp} - Template Implementation \textit{(if needed)}}
    \item {\texttt{component.cpp} - Implementation}
    \item {\texttt{component.t.cpp} - Unittest}
  \end{itemize}
  %
  This order also declares the file hierarchy, whereas deeper in the hierarchy
  means down in the above list.
\end{rule*}
%
\subsection{Relationships between the files of a Component }
\label{sec:relat-betw-files}

\subsubsection{Global Files}
\label{sec:global-files}

\begin{definition*}{Global Configuration File}
  \label{def:global-config-file}
  The configuration of all compilation units of \SYNEIGHT is controlled by
  one of the three global configuration files:
  \begin{itemize}
  \item {\tt <syneight/config/production.hpp>}
  \item {\tt <syneight/config/testsupport.hpp>}
  \item {\tt <syneight/config/unittest.hpp>}
  \end{itemize}
\end{definition*}
%
As should be clear now, production code does not know that
test-support or unittest code exists. Likewise, test-support code
knows nothing about unittest code. Thus these files form a hierarchy
in the following sense.
%
\begin{rule*}{Interface Hierarchy of Global Configuration Files}
  \label{rule:global-config-file-hier}
  The interface of {\tt <syneight/config/production.hpp>} must be
  subset of the interface of {\tt <syneight/config/testsupport.hpp>}
  and the interface of {\tt <syneight/config/testsupport.hpp>} must be a
  subset of the interface of {\tt <syneight/config/unittest.hpp>}.
\end{rule*}
%
\begin{rule*}{Global Configuration must be included first}
  \label{rule:global-config-first}
  Every source file of \SYNEIGHT must include one of these files before
  including any other file (being part of \SYNEIGHT or not).
  %
  The files included by the global configuration files are the only
  exception to this rule.\\
  %
  The rules following this definition will make sure that this
  constraint is matched.
\end{rule*}
%
\begin{rule*}{Files included into the global configuration files}
  A file which is included in a global configuration file, must not
  be included by a file which includes one of the global configuration
  files. 
\end{rule*}
%
The above rule divides the include files into two sets. First, there
are include files which are part of the configuration --  they are
included by one or more global configuration files. Second, there are
configured files which include the global configuration files.
%
\TODO{Add rules.}
%
\begin{remark*}{Test Support and Unittest change the configuration}
  Note that in the case of a unittest compilation unit, the first
  included file is {\tt <syneight/config/unittest.hpp>}. Later,
  ordinary \SYNEIGHT files are included, which include {\tt
    <syneight/config/production.hpp>}. 
  %
  It is important to note that the settings of the unittest
  configuration can override the settings of the production
  configuration\footnote{Such overriding must be exercised with care,
    since the code must be compatible with the usual production code.}
  and can expand the basic environment interface of \SYNEIGHT (such as
  providing additional annotations).
\end{remark*}

\subsubsection{Local relationships}

\begin{rule*}{Inclusion of \texttt{<component.hpp>}}
  \label{rule:inclusion-component-hpp}
  All files in a component need to \texttt{\#include <component.hpp>}.
\end{rule*}
%
This rule does not apply for \texttt{component.hpp} itself for obvious reasons.
As an advantage of this rule, we need not \texttt{\#include} the Global
Configuration Files in every single file, just in the Interface.

\subsection{Relationships between Components}

\begin{guideline*}{Where to \texttt {\#include} external components}
  The deeper in the hierarchy (see Rule \ref{rule:files-component})
  an \texttt {\#include} is made, the better.
\end{guideline*}

\subsection{Interface}
\label{sec:interface}

The Interface file is largely the header file, as \texttt{\#included} by the
user of a component.
%
Some simple guidelines are to be obeyed in coding this part.

\begin{guideline*}{Interface Programming Style}
  The {\tt component.hpp}, being the topmost file in the hierarchy,
  needs best documentation, rigid naming and consistent including of the
  dependencies.
\end{guideline*}

\begin{guideline*}{Contents of the Interface file}
  The interface file typically consists of the following:
  \begin{itemize}
    \item{\texttt{\#include <syneight/config/production.hpp>}}
    \item{Includes of all dependencies}
    \item{Named namespaces}
    \item{Declarations}
    \item{Type definitions}
    \item{Inline-function definitions}
    \item{Constant definitions}
  \end{itemize}
\end{guideline*}

\subsection{Implementation}
\label{sec:implementation}

The implementation is realized in two files. The first in the hierarchy
is the Template Implementation file. This makes it possible to (...)
\TODO{Perhaps \CS nows what exactly to write here...}
(...) The other file implements all the rest that isn't done in the Template
Implementation file. For both we set up the following guidelines.

\begin{guideline*}{Template Implementation Programming Style}
  The {\tt component.impl.hpp} is used only to implement templates and other
  code structures that need to be included directly by the user (or where
  he/she is better off with including them). Remember that a file is compiled
  each time it is included, and this file is read by the user directly, so
  keep the code in here neat and clean.
\end{guideline*}

\begin{guideline*}{Implementation Programming Style}
  The {\tt component.cpp} contains the main code of the component. All
  declarations in the Interface are defined here. A good style in general
  is mandatory but not as strict as in the other files.
\end{guideline*}
%
There are no predefined dependencies between the two mentioned implementation
files, so they will be kept independent by default.
%
\begin{guideline*}{Implementation / Template Implementation Contents}
  The implementation files typically consist of the following:
  \begin{itemize}
    \item{{\tt \#include <component.hpp>}}
    \item{Includes of all dependencies}
    \item{\TODO{Something missing here...}}
  \end{itemize}
\end{guideline*}
%
The implementation is largely in the hand of the component developer. However,
a few rules need to be obeyed. See Chapter \ref{cha:development-rules} for
details.

\subsection{Unittest}
\label{sec:unittest}

\section{Test Support Components}
\label{sec:test-supp-comp}






%%% Local Variables: 
%%% mode: latex
%%% TeX-master: "standards"
%%% End: 
% vim:ts=4:sw=4
