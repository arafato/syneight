%------------------------------------------------------------------------------
% Syneight - A soft-realtime transaction monitor.
% Copyright (C) 2003-2004 The Syneight Group.
%
% TODO: License.
%------------------------------------------------------------------------------

%%%%%%%%%%%%%%%%%%%%%%%%%%%%%%%%%%%%%%%%%%%%%%%%%%%%%%%%%%%%%%%%%%%%%%%%%%%%%%
%
% Meeting agenda and protocol 'environment'.
% 
% Author: cbs
% 
% This is a new meeting 'environment' which replaces the old one.
% 
% Here is an example of a agenda/protocol in the new format:
% 
% ----------------------------------------
% 
% %------------------------------------------------------------------------------
% Syneight - A soft-realtime transaction monitor.
% Copyright (C) 2003-2004 The Syneight Group.
%
% TODO: License.
%------------------------------------------------------------------------------

%%%%%%%%%%%%%%%%%%%%%%%%%%%%%%%%%%%%%%%%%%%%%%%%%%%%%%%%%%%%%%%%%%%%%%%%%%%%%%
%
% Meeting agenda and protocol 'environment'.
% 
% Author: cbs
% 
% This is a new meeting 'environment' which replaces the old one.
% 
% Here is an example of a agenda/protocol in the new format:
% 
% ----------------------------------------
% 
% %------------------------------------------------------------------------------
% Syneight - A soft-realtime transaction monitor.
% Copyright (C) 2003-2004 The Syneight Group.
%
% TODO: License.
%------------------------------------------------------------------------------

%%%%%%%%%%%%%%%%%%%%%%%%%%%%%%%%%%%%%%%%%%%%%%%%%%%%%%%%%%%%%%%%%%%%%%%%%%%%%%
%
% Meeting agenda and protocol 'environment'.
% 
% Author: cbs
% 
% This is a new meeting 'environment' which replaces the old one.
% 
% Here is an example of a agenda/protocol in the new format:
% 
% ----------------------------------------
% 
% %------------------------------------------------------------------------------
% Syneight - A soft-realtime transaction monitor.
% Copyright (C) 2003-2004 The Syneight Group.
%
% TODO: License.
%------------------------------------------------------------------------------

%%%%%%%%%%%%%%%%%%%%%%%%%%%%%%%%%%%%%%%%%%%%%%%%%%%%%%%%%%%%%%%%%%%%%%%%%%%%%%
%
% Meeting agenda and protocol 'environment'.
% 
% Author: cbs
% 
% This is a new meeting 'environment' which replaces the old one.
% 
% Here is an example of a agenda/protocol in the new format:
% 
% ----------------------------------------
% 
% \input{syneight/meeting.tex}
% 
% \begin{document}
% 
% \setTitle{Sample Meeting}
% 
% % set further title page information (omitted in this example)
% 
% \meetingtitlepage
% 
% \agenda
% 
% \topic{Topic 1}{10}
% 
% % topic content
% 
% \topic{Topic 2}{15}
% 
% % topic content
% 
% \protocol
% 
% \topic{Topic 1}{12}
% 
% % topic content
% 
% \topic{Topic 2}{14}
% 
% % topic content
% 
% \end{document}
% 
% ----------------------------------------
% 
% I think this is a little easier to use than the previous
% environment, since the document structure is now really simple.
% 
% Basically, I dropped all the "paperversion" stuff, thus removing the
% agenda/protocol summary. If anybody found those really useful, I'm
% sorry, but in my opinion, these are not really necessary. The agenda
% should not be more than one or two pages long anyway, so there is no
% real need for an extra summary, and the protocol does not need a
% summary at all.
% 
% The main advantage of the new environment is, that content within
% "topics" (formerly "issues", but the command \issue seemingly already
% exists somewhere in latex) can be any valid latex-code (e.g. a
% "verbatim" environment). This is because the content is no longer
% passed as an argument to the command. The structural commands are now
% more like "section"/"subsection" commands.
% 
% In contrast to the old environment, the content of topics is no longer
% restricted to a list of 'subissues', but can be anything the author
% wants to put there.
% 
% Topic headers are still numbered and contain time tracking information
% (start, stop and duration in minutes).
% 
% The title page now contains only a table with general information
% about the meeting (title, date, time, participants, ...). This
% information is set via \setXXX commands (see below) and then displayed
% with the \meetingtitlepage command.
% 
% I also added a command and environment for making notes: \note and
% 'longnote'-environment. These should be used for information that was
% not actually discussed during the meeting, but was added later to the
% protocol.
%
%%%%%%%%%%%%%%%%%%%%%%%%%%%%%%%%%%%%%%%%%%%%%%%%%%%%%%%%%%%%%%%%%%%%%%%%%%%%%%

\documentclass[a4paper, 10pt]{article}

\setcounter{secnumdepth}{3}

\usepackage{fullpage}
\usepackage{xspace}
\usepackage{varioref}
\usepackage{epic}
\usepackage{ecltree}
\usepackage{amsmath}
\usepackage{tabularx}
\usepackage{epsfig}
\usepackage{hhline}
\usepackage{multirow}
\usepackage{hhline}
\usepackage{calc}
\usepackage{ifthen}
\usepackage{graphics}

\sloppy

%%%%%%%%%%%%%%%%%%%%%%%%%%%%%%%%%%%%%%%%%%%%%%%%%%%%%%%%%%%%%%%%%%%%%%%%%%%%%%
%
% Counters used for topic numbering and time tracking.
%
%%%%%%%%%%%%%%%%%%%%%%%%%%%%%%%%%%%%%%%%%%%%%%%%%%%%%%%%%%%%%%%%%%%%%%%%%%%%%%

\newcounter{counttime}
\newcounter{counttopics}

%%%%%%%%%%%%%%%%%%%%%%%%%%%%%%%%%%%%%%%%%%%%%%%%%%%%%%%%%%%%%%%%%%%%%%%%%%%%%%
%
% Commands for setting title-page information
%
% These must be called before \meetingtitlepage is called!
%
%%%%%%%%%%%%%%%%%%%%%%%%%%%%%%%%%%%%%%%%%%%%%%%%%%%%%%%%%%%%%%%%%%%%%%%%%%%%%%

\newcommand{\setTitle}[1]{\def\meetingTitle{#1}}
\newcommand{\setCallers}[1]{\def\meetingCallers{#1}}
\newcommand{\setProposedDate}[1]{\def\meetingProposedDate{#1}}
\newcommand{\setProposedTime}[1]{\def\meetingProposedTime{#1}}
\newcommand{\setProposedAttendants}[1]{\def\meetingProposedAttendants{#1}}
\newcommand{\setProposedModerator}[1]{\def\meetingProposedModerator{#1}}
\newcommand{\setProposedProtocolWriter}[1]{\def\meetingProposedProtocolWriter{#1}}
\newcommand{\setDate}[1]{\def\meetingDate{#1}}
\newcommand{\setTime}[1]{\def\meetingTime{#1}}
\newcommand{\setAttendants}[1]{\def\meetingAttendants{#1}}
\newcommand{\setModerator}[1]{\def\meetingModerator{#1}}
\newcommand{\setProtocolWriter}[1]{\def\meetingProtocolWriter{#1}}

%%%%%%%%%%%%%%%%%%%%%%%%%%%%%%%%%%%%%%%%%%%%%%%%%%%%%%%%%%%%%%%%%%%%%%%%%%%%%%
%
% Command \heading{title} ... internal
%
% This is used to generate a section heading (on the titlepage and at the
% beginnings of the agenda and protocol).
%
%%%%%%%%%%%%%%%%%%%%%%%%%%%%%%%%%%%%%%%%%%%%%%%%%%%%%%%%%%%%%%%%%%%%%%%%%%%%%%

\newcommand{\heading}[1]{%
  \begin{center}%
  {\Large \bf #1}%
  \end{center}%
}

%%%%%%%%%%%%%%%%%%%%%%%%%%%%%%%%%%%%%%%%%%%%%%%%%%%%%%%%%%%%%%%%%%%%%%%%%%%%%%
%
% Command \meetingtitlepage
%
% Generates the title-page, which contains the information set with the
% commands above.
%
%%%%%%%%%%%%%%%%%%%%%%%%%%%%%%%%%%%%%%%%%%%%%%%%%%%%%%%%%%%%%%%%%%%%%%%%%%%%%%

\newcommand{\meetingtitlepage}{%
  \heading{Meeting: \meetingTitle}%
  \bigskip%
  \begin{tabular}{|p{30mm}|p{120mm}|} \hline%
    Callers         & \meetingCallers                \\ \hline%
  \end{tabular}%
  \bigskip\\%
  {\bf Proposed:}%
  \smallskip\\%
  \begin{tabular}{|p{30mm}|p{120mm}|} \hline%
    Date            & \meetingProposedDate           \\ \hline%
    Time            & \meetingProposedTime           \\ \hline%
    Participants    & \meetingProposedAttendants     \\ \hline%
    Moderator       & \meetingProposedModerator      \\ \hline%
    Protocol Writer & \meetingProposedProtocolWriter \\ \hline%
  \end{tabular}%
  \bigskip\\%
  {\bf Actual:}%
  \smallskip\\%
  \begin{tabular}{|p{30mm}|p{120mm}|} \hline%
    Date            & \meetingDate                   \\ \hline%
    Time            & \meetingTime                   \\ \hline%
    Participants    & \meetingAttendants             \\ \hline%
    Moderator       & \meetingModerator              \\ \hline%
    Protocol Writer & \meetingProtocolWriter         \\ \hline%
  \end{tabular}%
}

%%%%%%%%%%%%%%%%%%%%%%%%%%%%%%%%%%%%%%%%%%%%%%%%%%%%%%%%%%%%%%%%%%%%%%%%%%%%%%
%
% Command \newsection{title} ... internal
%
% Used to start a new section (agenda or protocol).
%
%%%%%%%%%%%%%%%%%%%%%%%%%%%%%%%%%%%%%%%%%%%%%%%%%%%%%%%%%%%%%%%%%%%%%%%%%%%%%%

\newcommand{\newsection}[1]{%
  \newpage%
  \setcounter{counttime}{0}%
  \setcounter{counttopics}{1}%
  \heading{#1}%
}

%%%%%%%%%%%%%%%%%%%%%%%%%%%%%%%%%%%%%%%%%%%%%%%%%%%%%%%%%%%%%%%%%%%%%%%%%%%%%%
%
% Command \agenda
%
% Starts the 'Agenda' section.
%
% All topic counters are reset.
%
%%%%%%%%%%%%%%%%%%%%%%%%%%%%%%%%%%%%%%%%%%%%%%%%%%%%%%%%%%%%%%%%%%%%%%%%%%%%%%

\newcommand{\agenda}{%
  \newsection{Agenda}%
}

%%%%%%%%%%%%%%%%%%%%%%%%%%%%%%%%%%%%%%%%%%%%%%%%%%%%%%%%%%%%%%%%%%%%%%%%%%%%%%
%
% Command \protocol
%
% Starts the 'Protocol' section.
%
% All topic counters are reset.
%
%%%%%%%%%%%%%%%%%%%%%%%%%%%%%%%%%%%%%%%%%%%%%%%%%%%%%%%%%%%%%%%%%%%%%%%%%%%%%%

\newcommand{\protocol}{%
  \newsection{Protocol}%
}

%%%%%%%%%%%%%%%%%%%%%%%%%%%%%%%%%%%%%%%%%%%%%%%%%%%%%%%%%%%%%%%%%%%%%%%%%%%%%%
%
% Command \topic{title}{duration}
%
% Starts a new topic.
%
% Arguments:
%
%   title    ... the topic title
%   duration ... the duration in minutes (projected duration for agenda, 
%                actual duration for protocol).
%
%%%%%%%%%%%%%%%%%%%%%%%%%%%%%%%%%%%%%%%%%%%%%%%%%%%%%%%%%%%%%%%%%%%%%%%%%%%%%%

\newcommand{\topic}[2]{%
  \bigskip%
  \noindent%
  \begin{tabular}{|p{7mm}|p{110mm}|p{35mm}|}%
  \hline%
  \hfill\thecounttopics & \textbf{#1} & \hfill #2 min (\thecounttime\addtocounter{counttime}{#2}--\thecounttime)%
  \\%
  \hline%
  \end{tabular}%
  \medskip%
  \addtocounter{counttopics}{1}%
}

%%%%%%%%%%%%%%%%%%%%%%%%%%%%%%%%%%%%%%%%%%%%%%%%%%%%%%%%%%%%%%%%%%%%%%%%%%%%%%
%
% Note commands:
%
%   \note{user}{text}
%
%   \begin{longnote}{user}
%      % text
%   \end{longnote}
%
% These should be used to add information to the protocol that was _not_
% discussed during the meeting (e.g. additional information or notions added
% by the protocol writer or a reviewer).
%
% Use \note for short (one sentence) notes and longnote-environment for
% longer notes (this creates a quote-environment around the note's content).
%
% Arguments:
%
%   user ... the name of the user that added the note.
%   text ... the note's content.
%
%%%%%%%%%%%%%%%%%%%%%%%%%%%%%%%%%%%%%%%%%%%%%%%%%%%%%%%%%%%%%%%%%%%%%%%%%%%%%%

\newcommand{\note}[2]{%
  [{\bf Note by #1:} #2]%
}

\newenvironment{longnote}[1]{%
  {\bf Note by #1:}%
  \begin{quote}%
}{%
  \end{quote}%
}

%%%%%%%%%%%%%%%%%%%%%%%%%%%%%%%%%%%%%%%%%%%%%%%%%%%%%%%%%%%%%%%%%%%%%%%%%%%%%%
%
% Include standard files
%
%%%%%%%%%%%%%%%%%%%%%%%%%%%%%%%%%%%%%%%%%%%%%%%%%%%%%%%%%%%%%%%%%%%%%%%%%%%%%%

\input{syneight/people.tex}
\input{syneight/abbreviations.tex}

% vim:ts=4:sw=4

% 
% \begin{document}
% 
% \setTitle{Sample Meeting}
% 
% % set further title page information (omitted in this example)
% 
% \meetingtitlepage
% 
% \agenda
% 
% \topic{Topic 1}{10}
% 
% % topic content
% 
% \topic{Topic 2}{15}
% 
% % topic content
% 
% \protocol
% 
% \topic{Topic 1}{12}
% 
% % topic content
% 
% \topic{Topic 2}{14}
% 
% % topic content
% 
% \end{document}
% 
% ----------------------------------------
% 
% I think this is a little easier to use than the previous
% environment, since the document structure is now really simple.
% 
% Basically, I dropped all the "paperversion" stuff, thus removing the
% agenda/protocol summary. If anybody found those really useful, I'm
% sorry, but in my opinion, these are not really necessary. The agenda
% should not be more than one or two pages long anyway, so there is no
% real need for an extra summary, and the protocol does not need a
% summary at all.
% 
% The main advantage of the new environment is, that content within
% "topics" (formerly "issues", but the command \issue seemingly already
% exists somewhere in latex) can be any valid latex-code (e.g. a
% "verbatim" environment). This is because the content is no longer
% passed as an argument to the command. The structural commands are now
% more like "section"/"subsection" commands.
% 
% In contrast to the old environment, the content of topics is no longer
% restricted to a list of 'subissues', but can be anything the author
% wants to put there.
% 
% Topic headers are still numbered and contain time tracking information
% (start, stop and duration in minutes).
% 
% The title page now contains only a table with general information
% about the meeting (title, date, time, participants, ...). This
% information is set via \setXXX commands (see below) and then displayed
% with the \meetingtitlepage command.
% 
% I also added a command and environment for making notes: \note and
% 'longnote'-environment. These should be used for information that was
% not actually discussed during the meeting, but was added later to the
% protocol.
%
%%%%%%%%%%%%%%%%%%%%%%%%%%%%%%%%%%%%%%%%%%%%%%%%%%%%%%%%%%%%%%%%%%%%%%%%%%%%%%

\documentclass[a4paper, 10pt]{article}

\setcounter{secnumdepth}{3}

\usepackage{fullpage}
\usepackage{xspace}
\usepackage{varioref}
\usepackage{epic}
\usepackage{ecltree}
\usepackage{amsmath}
\usepackage{tabularx}
\usepackage{epsfig}
\usepackage{hhline}
\usepackage{multirow}
\usepackage{hhline}
\usepackage{calc}
\usepackage{ifthen}
\usepackage{graphics}

\sloppy

%%%%%%%%%%%%%%%%%%%%%%%%%%%%%%%%%%%%%%%%%%%%%%%%%%%%%%%%%%%%%%%%%%%%%%%%%%%%%%
%
% Counters used for topic numbering and time tracking.
%
%%%%%%%%%%%%%%%%%%%%%%%%%%%%%%%%%%%%%%%%%%%%%%%%%%%%%%%%%%%%%%%%%%%%%%%%%%%%%%

\newcounter{counttime}
\newcounter{counttopics}

%%%%%%%%%%%%%%%%%%%%%%%%%%%%%%%%%%%%%%%%%%%%%%%%%%%%%%%%%%%%%%%%%%%%%%%%%%%%%%
%
% Commands for setting title-page information
%
% These must be called before \meetingtitlepage is called!
%
%%%%%%%%%%%%%%%%%%%%%%%%%%%%%%%%%%%%%%%%%%%%%%%%%%%%%%%%%%%%%%%%%%%%%%%%%%%%%%

\newcommand{\setTitle}[1]{\def\meetingTitle{#1}}
\newcommand{\setCallers}[1]{\def\meetingCallers{#1}}
\newcommand{\setProposedDate}[1]{\def\meetingProposedDate{#1}}
\newcommand{\setProposedTime}[1]{\def\meetingProposedTime{#1}}
\newcommand{\setProposedAttendants}[1]{\def\meetingProposedAttendants{#1}}
\newcommand{\setProposedModerator}[1]{\def\meetingProposedModerator{#1}}
\newcommand{\setProposedProtocolWriter}[1]{\def\meetingProposedProtocolWriter{#1}}
\newcommand{\setDate}[1]{\def\meetingDate{#1}}
\newcommand{\setTime}[1]{\def\meetingTime{#1}}
\newcommand{\setAttendants}[1]{\def\meetingAttendants{#1}}
\newcommand{\setModerator}[1]{\def\meetingModerator{#1}}
\newcommand{\setProtocolWriter}[1]{\def\meetingProtocolWriter{#1}}

%%%%%%%%%%%%%%%%%%%%%%%%%%%%%%%%%%%%%%%%%%%%%%%%%%%%%%%%%%%%%%%%%%%%%%%%%%%%%%
%
% Command \heading{title} ... internal
%
% This is used to generate a section heading (on the titlepage and at the
% beginnings of the agenda and protocol).
%
%%%%%%%%%%%%%%%%%%%%%%%%%%%%%%%%%%%%%%%%%%%%%%%%%%%%%%%%%%%%%%%%%%%%%%%%%%%%%%

\newcommand{\heading}[1]{%
  \begin{center}%
  {\Large \bf #1}%
  \end{center}%
}

%%%%%%%%%%%%%%%%%%%%%%%%%%%%%%%%%%%%%%%%%%%%%%%%%%%%%%%%%%%%%%%%%%%%%%%%%%%%%%
%
% Command \meetingtitlepage
%
% Generates the title-page, which contains the information set with the
% commands above.
%
%%%%%%%%%%%%%%%%%%%%%%%%%%%%%%%%%%%%%%%%%%%%%%%%%%%%%%%%%%%%%%%%%%%%%%%%%%%%%%

\newcommand{\meetingtitlepage}{%
  \heading{Meeting: \meetingTitle}%
  \bigskip%
  \begin{tabular}{|p{30mm}|p{120mm}|} \hline%
    Callers         & \meetingCallers                \\ \hline%
  \end{tabular}%
  \bigskip\\%
  {\bf Proposed:}%
  \smallskip\\%
  \begin{tabular}{|p{30mm}|p{120mm}|} \hline%
    Date            & \meetingProposedDate           \\ \hline%
    Time            & \meetingProposedTime           \\ \hline%
    Participants    & \meetingProposedAttendants     \\ \hline%
    Moderator       & \meetingProposedModerator      \\ \hline%
    Protocol Writer & \meetingProposedProtocolWriter \\ \hline%
  \end{tabular}%
  \bigskip\\%
  {\bf Actual:}%
  \smallskip\\%
  \begin{tabular}{|p{30mm}|p{120mm}|} \hline%
    Date            & \meetingDate                   \\ \hline%
    Time            & \meetingTime                   \\ \hline%
    Participants    & \meetingAttendants             \\ \hline%
    Moderator       & \meetingModerator              \\ \hline%
    Protocol Writer & \meetingProtocolWriter         \\ \hline%
  \end{tabular}%
}

%%%%%%%%%%%%%%%%%%%%%%%%%%%%%%%%%%%%%%%%%%%%%%%%%%%%%%%%%%%%%%%%%%%%%%%%%%%%%%
%
% Command \newsection{title} ... internal
%
% Used to start a new section (agenda or protocol).
%
%%%%%%%%%%%%%%%%%%%%%%%%%%%%%%%%%%%%%%%%%%%%%%%%%%%%%%%%%%%%%%%%%%%%%%%%%%%%%%

\newcommand{\newsection}[1]{%
  \newpage%
  \setcounter{counttime}{0}%
  \setcounter{counttopics}{1}%
  \heading{#1}%
}

%%%%%%%%%%%%%%%%%%%%%%%%%%%%%%%%%%%%%%%%%%%%%%%%%%%%%%%%%%%%%%%%%%%%%%%%%%%%%%
%
% Command \agenda
%
% Starts the 'Agenda' section.
%
% All topic counters are reset.
%
%%%%%%%%%%%%%%%%%%%%%%%%%%%%%%%%%%%%%%%%%%%%%%%%%%%%%%%%%%%%%%%%%%%%%%%%%%%%%%

\newcommand{\agenda}{%
  \newsection{Agenda}%
}

%%%%%%%%%%%%%%%%%%%%%%%%%%%%%%%%%%%%%%%%%%%%%%%%%%%%%%%%%%%%%%%%%%%%%%%%%%%%%%
%
% Command \protocol
%
% Starts the 'Protocol' section.
%
% All topic counters are reset.
%
%%%%%%%%%%%%%%%%%%%%%%%%%%%%%%%%%%%%%%%%%%%%%%%%%%%%%%%%%%%%%%%%%%%%%%%%%%%%%%

\newcommand{\protocol}{%
  \newsection{Protocol}%
}

%%%%%%%%%%%%%%%%%%%%%%%%%%%%%%%%%%%%%%%%%%%%%%%%%%%%%%%%%%%%%%%%%%%%%%%%%%%%%%
%
% Command \topic{title}{duration}
%
% Starts a new topic.
%
% Arguments:
%
%   title    ... the topic title
%   duration ... the duration in minutes (projected duration for agenda, 
%                actual duration for protocol).
%
%%%%%%%%%%%%%%%%%%%%%%%%%%%%%%%%%%%%%%%%%%%%%%%%%%%%%%%%%%%%%%%%%%%%%%%%%%%%%%

\newcommand{\topic}[2]{%
  \bigskip%
  \noindent%
  \begin{tabular}{|p{7mm}|p{110mm}|p{35mm}|}%
  \hline%
  \hfill\thecounttopics & \textbf{#1} & \hfill #2 min (\thecounttime\addtocounter{counttime}{#2}--\thecounttime)%
  \\%
  \hline%
  \end{tabular}%
  \medskip%
  \addtocounter{counttopics}{1}%
}

%%%%%%%%%%%%%%%%%%%%%%%%%%%%%%%%%%%%%%%%%%%%%%%%%%%%%%%%%%%%%%%%%%%%%%%%%%%%%%
%
% Note commands:
%
%   \note{user}{text}
%
%   \begin{longnote}{user}
%      % text
%   \end{longnote}
%
% These should be used to add information to the protocol that was _not_
% discussed during the meeting (e.g. additional information or notions added
% by the protocol writer or a reviewer).
%
% Use \note for short (one sentence) notes and longnote-environment for
% longer notes (this creates a quote-environment around the note's content).
%
% Arguments:
%
%   user ... the name of the user that added the note.
%   text ... the note's content.
%
%%%%%%%%%%%%%%%%%%%%%%%%%%%%%%%%%%%%%%%%%%%%%%%%%%%%%%%%%%%%%%%%%%%%%%%%%%%%%%

\newcommand{\note}[2]{%
  [{\bf Note by #1:} #2]%
}

\newenvironment{longnote}[1]{%
  {\bf Note by #1:}%
  \begin{quote}%
}{%
  \end{quote}%
}

%%%%%%%%%%%%%%%%%%%%%%%%%%%%%%%%%%%%%%%%%%%%%%%%%%%%%%%%%%%%%%%%%%%%%%%%%%%%%%
%
% Include standard files
%
%%%%%%%%%%%%%%%%%%%%%%%%%%%%%%%%%%%%%%%%%%%%%%%%%%%%%%%%%%%%%%%%%%%%%%%%%%%%%%

%------------------------------------------------------------------------------
% Syneight - A soft-realtime transaction monitor.
% Copyright (C) 2003-2004 The Syneight Group.
%
% TODO: License.
%------------------------------------------------------------------------------

\newcommand{\CS}{\textsc{Christian}\xspace}
\newcommand{\PET}{\textsc{Piotr}\xspace}
\newcommand{\OLLI}{\textsc{Olli}\xspace}
\newcommand{\TOBI}{\textsc{Tobi}\xspace}
\newcommand{\UWE}{\textsc{Uwe}\xspace}
\newcommand{\DHR}{\textsc{Daniel}\xspace}
\newcommand{\ANDREAS}{\textsc{Andreas}\xspace}

\newcommand{\TEAM}{\CS \PET \OLLI \TOBI \UWE \DHR \ANDREAS}

%%% Local Variables: 
%%% mode: latex
%%% TeX-master: t
%%% End: 
% vim:ts=4:sw=4

%------------------------------------------------------------------------------
% Syneight - A soft-realtime transaction monitor.
% Copyright (C) 2003-2004 The Syneight Group.
%
% TODO: License.
%------------------------------------------------------------------------------

\usepackage{xspace}

\newcommand{\SYNEIGHT}{\textsc{Syneight}\xspace}
\newcommand{\DSM}{\textsc{Dsm}\xspace}

\newcommand{\MMOG}{\textsc{Mmog}\xspace}
\newcommand{\MMOGS}{\textsc{Mmog}s\xspace}
\newcommand{\MMORG}{\textsc{Mmorg}\xspace}
\newcommand{\MMORGS}{\textsc{Mmorg}s\xspace}
\newcommand{\NVE}{\textsc{Nve}\xspace}
\newcommand{\NVES}{\textsc{Nve}s\xspace}

\newcommand{\DBAI}{{ \normalfont\textsc{dbai}}\xspace}

\newcommand{\EmeLevel}{{\tt EME\_LEVEL}\xspace}
\newcommand{\AleLevel}{{\tt ALE\_LEVEL}\xspace}
\newcommand{\ProLevel}{{\tt PRO\_LEVEL}\xspace}
\newcommand{\DebLevel}{{\tt DEB\_LEVEL}\xspace}
\newcommand{\AudLevel}{{\tt AUD\_LEVEL}\xspace}
\newcommand{\TesLevel}{{\tt TES\_LEVEL}\xspace}
\newcommand{\SysLevel}{{\tt SYS\_LEVEL}\xspace}

\newcommand{\SwitchTestAnnotations}{{\tt SYNEIGHT\_\_SWITCH\_\_TEST\_ANNOTATIONS}\xspace}
\newcommand{\SwitchProductionAnnotations}{{\tt SYNEIGHT\_\_SWITCH\_\_PRODUCTION\_ANNOTATIONS}\xspace}
\newcommand{\SwitchLogProChecks}{{\tt SYNEIGHT\_\_SWITCH\_\_LOG\_PRO\_CHECKS}\xspace}
\newcommand{\SwitchBuildLevel}{{\tt SYNEIGHT\_\_SWITCH\_\_BUILD\_LEVEL}\xspace}
\newcommand{\SwitchInline}{{\tt SYNEIGHT\_\_SWITCH\_\_INLINE}\xspace}
\newcommand{\BuildLevel}{{\tt SYNEIGHT\_\_BUILD\_LEVEL}\xspace}
\newcommand{\BuildLevelType}{{\tt Build\_Level\_t}\xspace}
\newcommand{\VerbosityLevelType}{{\tt Verbosity\_Level\_t}\xspace}

\newcommand{\InternalParamBaseFile}{{\tt SYNEIGHT\_\_INTERNAL\_\_PARAM\_\_BASE\_FILE}\xspace}
\newcommand{\InternalParamPrettyFunction}{{\tt SYNEIGHT\_\_INTERNAL\_\_PARAM\_\_PRETTY\_FUNCTION}\xspace}

\newcommand{\InternalParamBaseExceptionType}{{\tt SYNEIGHT\_\_INTERNAL\_\_PARAM\_\_BASE\_EXCEPTION\_TYPE}\xspace}
\newcommand{\InternalParamTestExceptionType}{{\tt SYNEIGHT\_\_INTERNAL\_\_PARAM\_\_TEST\_EXCEPTION\_TYPE}\xspace}
\newcommand{\InternalParamExceptionStrWhat}{{\tt SYNEIGHT\_\_INTERNAL\_\_PARAM\_\_EXCEPTION\_WHAT}\xspace}
\newcommand{\InternalParamTestExceptionStrWhat}{{\tt SYNEIGHT\_\_INTERNAL\_\_PARAM\_\_TEST\_\_EXCEPTION\_WHAT}\xspace}
\newcommand{\InternalSwitchHandleTestExceptionExplicitly}{{\tt SYNEIGHT\_\_INTERNAL\_\_SWITCH\_\_HANDLE\_TEST\_EXCEPTION\_EXPLICITLY}\xspace}

\newcommand{\InternalParamExceptionA}{{\tt SYNEIGHT\_\_INTERNAL\_\_PARAM\_\_EXCEPTION0}\xspace}
\newcommand{\InternalParamExceptionB}{{\tt SYNEIGHT\_\_INTERNAL\_\_PARAM\_\_EXCEPTION1}\xspace}
\newcommand{\InternalParamExceptionC}{{\tt SYNEIGHT\_\_INTERNAL\_\_PARAM\_\_EXCEPTION2}\xspace}
\newcommand{\InternalParamExceptionD}{{\tt SYNEIGHT\_\_INTERNAL\_\_PARAM\_\_EXCEPTION3}\xspace}
\newcommand{\InternalParamExceptionE}{{\tt SYNEIGHT\_\_INTERNAL\_\_PARAM\_\_EXCEPTION4}\xspace}

\newcommand{\InternalParamThrowActionA}{{\tt SYNEIGHT\_\_INTERNAL\_\_PARAM\_\_THROW\_\_ACTION0}\xspace}
\newcommand{\InternalParamThrowActionB}{{\tt SYNEIGHT\_\_INTERNAL\_\_PARAM\_\_THROW\_\_ACTION1}\xspace}
\newcommand{\InternalParamThrowActionC}{{\tt SYNEIGHT\_\_INTERNAL\_\_PARAM\_\_THROW\_\_ACTION2}\xspace}
\newcommand{\InternalParamThrowActionD}{{\tt SYNEIGHT\_\_INTERNAL\_\_PARAM\_\_THROW\_\_ACTION3}\xspace}
\newcommand{\InternalParamThrowActionE}{{\tt SYNEIGHT\_\_INTERNAL\_\_PARAM\_\_THROW\_\_ACTION4}\xspace}

\newcommand{\InternalParamLogActionA}{{\tt SYNEIGHT\_\_INTERNAL\_\_PARAM\_\_LOG\_\_ACTION0}\xspace}
\newcommand{\InternalParamLogActionB}{{\tt SYNEIGHT\_\_INTERNAL\_\_PARAM\_\_LOG\_\_ACTION1}\xspace}
\newcommand{\InternalParamLogActionC}{{\tt SYNEIGHT\_\_INTERNAL\_\_PARAM\_\_LOG\_\_ACTION2}\xspace}
\newcommand{\InternalParamLogActionD}{{\tt SYNEIGHT\_\_INTERNAL\_\_PARAM\_\_LOG\_\_ACTION3}\xspace}
\newcommand{\InternalParamLogActionE}{{\tt SYNEIGHT\_\_INTERNAL\_\_PARAM\_\_LOG\_\_ACTION4}\xspace}


\newcommand{\XXXTrace}{{\tt SYNEIGHT\_\_XXX\_\_TRACE}\xspace}
\newcommand{\XXXBinaryTrace}{{\tt SYNEIGHT\_\_XXX\_\_BINARY\_TRACE}\xspace}

\newcommand{\XXXBlockEnter}{{\tt SYNEIGHT\_\_XXX\_\_BLOCK\_ENTER}\xspace}
\newcommand{\XXXBlockExit}{{\tt SYNEIGHT\_\_XXX\_\_BLOCK\_EXIT}\xspace}
\newcommand{\XXXBlockGuard}{{\tt SYNEIGHT\_\_XXX\_\_BLOCK\_GUARD}\xspace}

\newcommand{\XXXProcedureEnter}{{\tt SYNEIGHT\_\_XXX\_\_PROCEDURE\_ENTER}\xspace}
\newcommand{\XXXProcedureExit}{{\tt SYNEIGHT\_\_XXX\_\_PROCEDURE\_EXIT}\xspace}
\newcommand{\XXXProcedureGuard}{{\tt SYNEIGHT\_\_XXX\_\_PROCEDURE\_GUARD}\xspace}

\newcommand{\XXXMethodEnter}{{\tt SYNEIGHT\_\_XXX\_\_METHOD\_ENTER}\xspace}
\newcommand{\XXXMethodExit}{{\tt SYNEIGHT\_\_XXX\_\_METHOD\_EXIT}\xspace}
\newcommand{\XXXMethodGuard}{{\tt SYNEIGHT\_\_XXX\_\_METHOD\_GUARD}\xspace}



\newcommand{\Inline}{{\tt SYNEIGHT\_\_INLINE}\xspace}

%%% Local Variables: 
%%% mode: latex
%%% TeX-master: t
%%% End: 
% vim:ts=4:sw=4


% vim:ts=4:sw=4

% 
% \begin{document}
% 
% \setTitle{Sample Meeting}
% 
% % set further title page information (omitted in this example)
% 
% \meetingtitlepage
% 
% \agenda
% 
% \topic{Topic 1}{10}
% 
% % topic content
% 
% \topic{Topic 2}{15}
% 
% % topic content
% 
% \protocol
% 
% \topic{Topic 1}{12}
% 
% % topic content
% 
% \topic{Topic 2}{14}
% 
% % topic content
% 
% \end{document}
% 
% ----------------------------------------
% 
% I think this is a little easier to use than the previous
% environment, since the document structure is now really simple.
% 
% Basically, I dropped all the "paperversion" stuff, thus removing the
% agenda/protocol summary. If anybody found those really useful, I'm
% sorry, but in my opinion, these are not really necessary. The agenda
% should not be more than one or two pages long anyway, so there is no
% real need for an extra summary, and the protocol does not need a
% summary at all.
% 
% The main advantage of the new environment is, that content within
% "topics" (formerly "issues", but the command \issue seemingly already
% exists somewhere in latex) can be any valid latex-code (e.g. a
% "verbatim" environment). This is because the content is no longer
% passed as an argument to the command. The structural commands are now
% more like "section"/"subsection" commands.
% 
% In contrast to the old environment, the content of topics is no longer
% restricted to a list of 'subissues', but can be anything the author
% wants to put there.
% 
% Topic headers are still numbered and contain time tracking information
% (start, stop and duration in minutes).
% 
% The title page now contains only a table with general information
% about the meeting (title, date, time, participants, ...). This
% information is set via \setXXX commands (see below) and then displayed
% with the \meetingtitlepage command.
% 
% I also added a command and environment for making notes: \note and
% 'longnote'-environment. These should be used for information that was
% not actually discussed during the meeting, but was added later to the
% protocol.
%
%%%%%%%%%%%%%%%%%%%%%%%%%%%%%%%%%%%%%%%%%%%%%%%%%%%%%%%%%%%%%%%%%%%%%%%%%%%%%%

\documentclass[a4paper, 10pt]{article}

\setcounter{secnumdepth}{3}

\usepackage{fullpage}
\usepackage{xspace}
\usepackage{varioref}
\usepackage{epic}
\usepackage{ecltree}
\usepackage{amsmath}
\usepackage{tabularx}
\usepackage{epsfig}
\usepackage{hhline}
\usepackage{multirow}
\usepackage{hhline}
\usepackage{calc}
\usepackage{ifthen}
\usepackage{graphics}

\sloppy

%%%%%%%%%%%%%%%%%%%%%%%%%%%%%%%%%%%%%%%%%%%%%%%%%%%%%%%%%%%%%%%%%%%%%%%%%%%%%%
%
% Counters used for topic numbering and time tracking.
%
%%%%%%%%%%%%%%%%%%%%%%%%%%%%%%%%%%%%%%%%%%%%%%%%%%%%%%%%%%%%%%%%%%%%%%%%%%%%%%

\newcounter{counttime}
\newcounter{counttopics}

%%%%%%%%%%%%%%%%%%%%%%%%%%%%%%%%%%%%%%%%%%%%%%%%%%%%%%%%%%%%%%%%%%%%%%%%%%%%%%
%
% Commands for setting title-page information
%
% These must be called before \meetingtitlepage is called!
%
%%%%%%%%%%%%%%%%%%%%%%%%%%%%%%%%%%%%%%%%%%%%%%%%%%%%%%%%%%%%%%%%%%%%%%%%%%%%%%

\newcommand{\setTitle}[1]{\def\meetingTitle{#1}}
\newcommand{\setCallers}[1]{\def\meetingCallers{#1}}
\newcommand{\setProposedDate}[1]{\def\meetingProposedDate{#1}}
\newcommand{\setProposedTime}[1]{\def\meetingProposedTime{#1}}
\newcommand{\setProposedAttendants}[1]{\def\meetingProposedAttendants{#1}}
\newcommand{\setProposedModerator}[1]{\def\meetingProposedModerator{#1}}
\newcommand{\setProposedProtocolWriter}[1]{\def\meetingProposedProtocolWriter{#1}}
\newcommand{\setDate}[1]{\def\meetingDate{#1}}
\newcommand{\setTime}[1]{\def\meetingTime{#1}}
\newcommand{\setAttendants}[1]{\def\meetingAttendants{#1}}
\newcommand{\setModerator}[1]{\def\meetingModerator{#1}}
\newcommand{\setProtocolWriter}[1]{\def\meetingProtocolWriter{#1}}

%%%%%%%%%%%%%%%%%%%%%%%%%%%%%%%%%%%%%%%%%%%%%%%%%%%%%%%%%%%%%%%%%%%%%%%%%%%%%%
%
% Command \heading{title} ... internal
%
% This is used to generate a section heading (on the titlepage and at the
% beginnings of the agenda and protocol).
%
%%%%%%%%%%%%%%%%%%%%%%%%%%%%%%%%%%%%%%%%%%%%%%%%%%%%%%%%%%%%%%%%%%%%%%%%%%%%%%

\newcommand{\heading}[1]{%
  \begin{center}%
  {\Large \bf #1}%
  \end{center}%
}

%%%%%%%%%%%%%%%%%%%%%%%%%%%%%%%%%%%%%%%%%%%%%%%%%%%%%%%%%%%%%%%%%%%%%%%%%%%%%%
%
% Command \meetingtitlepage
%
% Generates the title-page, which contains the information set with the
% commands above.
%
%%%%%%%%%%%%%%%%%%%%%%%%%%%%%%%%%%%%%%%%%%%%%%%%%%%%%%%%%%%%%%%%%%%%%%%%%%%%%%

\newcommand{\meetingtitlepage}{%
  \heading{Meeting: \meetingTitle}%
  \bigskip%
  \begin{tabular}{|p{30mm}|p{120mm}|} \hline%
    Callers         & \meetingCallers                \\ \hline%
  \end{tabular}%
  \bigskip\\%
  {\bf Proposed:}%
  \smallskip\\%
  \begin{tabular}{|p{30mm}|p{120mm}|} \hline%
    Date            & \meetingProposedDate           \\ \hline%
    Time            & \meetingProposedTime           \\ \hline%
    Participants    & \meetingProposedAttendants     \\ \hline%
    Moderator       & \meetingProposedModerator      \\ \hline%
    Protocol Writer & \meetingProposedProtocolWriter \\ \hline%
  \end{tabular}%
  \bigskip\\%
  {\bf Actual:}%
  \smallskip\\%
  \begin{tabular}{|p{30mm}|p{120mm}|} \hline%
    Date            & \meetingDate                   \\ \hline%
    Time            & \meetingTime                   \\ \hline%
    Participants    & \meetingAttendants             \\ \hline%
    Moderator       & \meetingModerator              \\ \hline%
    Protocol Writer & \meetingProtocolWriter         \\ \hline%
  \end{tabular}%
}

%%%%%%%%%%%%%%%%%%%%%%%%%%%%%%%%%%%%%%%%%%%%%%%%%%%%%%%%%%%%%%%%%%%%%%%%%%%%%%
%
% Command \newsection{title} ... internal
%
% Used to start a new section (agenda or protocol).
%
%%%%%%%%%%%%%%%%%%%%%%%%%%%%%%%%%%%%%%%%%%%%%%%%%%%%%%%%%%%%%%%%%%%%%%%%%%%%%%

\newcommand{\newsection}[1]{%
  \newpage%
  \setcounter{counttime}{0}%
  \setcounter{counttopics}{1}%
  \heading{#1}%
}

%%%%%%%%%%%%%%%%%%%%%%%%%%%%%%%%%%%%%%%%%%%%%%%%%%%%%%%%%%%%%%%%%%%%%%%%%%%%%%
%
% Command \agenda
%
% Starts the 'Agenda' section.
%
% All topic counters are reset.
%
%%%%%%%%%%%%%%%%%%%%%%%%%%%%%%%%%%%%%%%%%%%%%%%%%%%%%%%%%%%%%%%%%%%%%%%%%%%%%%

\newcommand{\agenda}{%
  \newsection{Agenda}%
}

%%%%%%%%%%%%%%%%%%%%%%%%%%%%%%%%%%%%%%%%%%%%%%%%%%%%%%%%%%%%%%%%%%%%%%%%%%%%%%
%
% Command \protocol
%
% Starts the 'Protocol' section.
%
% All topic counters are reset.
%
%%%%%%%%%%%%%%%%%%%%%%%%%%%%%%%%%%%%%%%%%%%%%%%%%%%%%%%%%%%%%%%%%%%%%%%%%%%%%%

\newcommand{\protocol}{%
  \newsection{Protocol}%
}

%%%%%%%%%%%%%%%%%%%%%%%%%%%%%%%%%%%%%%%%%%%%%%%%%%%%%%%%%%%%%%%%%%%%%%%%%%%%%%
%
% Command \topic{title}{duration}
%
% Starts a new topic.
%
% Arguments:
%
%   title    ... the topic title
%   duration ... the duration in minutes (projected duration for agenda, 
%                actual duration for protocol).
%
%%%%%%%%%%%%%%%%%%%%%%%%%%%%%%%%%%%%%%%%%%%%%%%%%%%%%%%%%%%%%%%%%%%%%%%%%%%%%%

\newcommand{\topic}[2]{%
  \bigskip%
  \noindent%
  \begin{tabular}{|p{7mm}|p{110mm}|p{35mm}|}%
  \hline%
  \hfill\thecounttopics & \textbf{#1} & \hfill #2 min (\thecounttime\addtocounter{counttime}{#2}--\thecounttime)%
  \\%
  \hline%
  \end{tabular}%
  \medskip%
  \addtocounter{counttopics}{1}%
}

%%%%%%%%%%%%%%%%%%%%%%%%%%%%%%%%%%%%%%%%%%%%%%%%%%%%%%%%%%%%%%%%%%%%%%%%%%%%%%
%
% Note commands:
%
%   \note{user}{text}
%
%   \begin{longnote}{user}
%      % text
%   \end{longnote}
%
% These should be used to add information to the protocol that was _not_
% discussed during the meeting (e.g. additional information or notions added
% by the protocol writer or a reviewer).
%
% Use \note for short (one sentence) notes and longnote-environment for
% longer notes (this creates a quote-environment around the note's content).
%
% Arguments:
%
%   user ... the name of the user that added the note.
%   text ... the note's content.
%
%%%%%%%%%%%%%%%%%%%%%%%%%%%%%%%%%%%%%%%%%%%%%%%%%%%%%%%%%%%%%%%%%%%%%%%%%%%%%%

\newcommand{\note}[2]{%
  [{\bf Note by #1:} #2]%
}

\newenvironment{longnote}[1]{%
  {\bf Note by #1:}%
  \begin{quote}%
}{%
  \end{quote}%
}

%%%%%%%%%%%%%%%%%%%%%%%%%%%%%%%%%%%%%%%%%%%%%%%%%%%%%%%%%%%%%%%%%%%%%%%%%%%%%%
%
% Include standard files
%
%%%%%%%%%%%%%%%%%%%%%%%%%%%%%%%%%%%%%%%%%%%%%%%%%%%%%%%%%%%%%%%%%%%%%%%%%%%%%%

%------------------------------------------------------------------------------
% Syneight - A soft-realtime transaction monitor.
% Copyright (C) 2003-2004 The Syneight Group.
%
% TODO: License.
%------------------------------------------------------------------------------

\newcommand{\CS}{\textsc{Christian}\xspace}
\newcommand{\PET}{\textsc{Piotr}\xspace}
\newcommand{\OLLI}{\textsc{Olli}\xspace}
\newcommand{\TOBI}{\textsc{Tobi}\xspace}
\newcommand{\UWE}{\textsc{Uwe}\xspace}
\newcommand{\DHR}{\textsc{Daniel}\xspace}
\newcommand{\ANDREAS}{\textsc{Andreas}\xspace}

\newcommand{\TEAM}{\CS \PET \OLLI \TOBI \UWE \DHR \ANDREAS}

%%% Local Variables: 
%%% mode: latex
%%% TeX-master: t
%%% End: 
% vim:ts=4:sw=4

%------------------------------------------------------------------------------
% Syneight - A soft-realtime transaction monitor.
% Copyright (C) 2003-2004 The Syneight Group.
%
% TODO: License.
%------------------------------------------------------------------------------

\usepackage{xspace}

\newcommand{\SYNEIGHT}{\textsc{Syneight}\xspace}
\newcommand{\DSM}{\textsc{Dsm}\xspace}

\newcommand{\MMOG}{\textsc{Mmog}\xspace}
\newcommand{\MMOGS}{\textsc{Mmog}s\xspace}
\newcommand{\MMORG}{\textsc{Mmorg}\xspace}
\newcommand{\MMORGS}{\textsc{Mmorg}s\xspace}
\newcommand{\NVE}{\textsc{Nve}\xspace}
\newcommand{\NVES}{\textsc{Nve}s\xspace}

\newcommand{\DBAI}{{ \normalfont\textsc{dbai}}\xspace}

\newcommand{\EmeLevel}{{\tt EME\_LEVEL}\xspace}
\newcommand{\AleLevel}{{\tt ALE\_LEVEL}\xspace}
\newcommand{\ProLevel}{{\tt PRO\_LEVEL}\xspace}
\newcommand{\DebLevel}{{\tt DEB\_LEVEL}\xspace}
\newcommand{\AudLevel}{{\tt AUD\_LEVEL}\xspace}
\newcommand{\TesLevel}{{\tt TES\_LEVEL}\xspace}
\newcommand{\SysLevel}{{\tt SYS\_LEVEL}\xspace}

\newcommand{\SwitchTestAnnotations}{{\tt SYNEIGHT\_\_SWITCH\_\_TEST\_ANNOTATIONS}\xspace}
\newcommand{\SwitchProductionAnnotations}{{\tt SYNEIGHT\_\_SWITCH\_\_PRODUCTION\_ANNOTATIONS}\xspace}
\newcommand{\SwitchLogProChecks}{{\tt SYNEIGHT\_\_SWITCH\_\_LOG\_PRO\_CHECKS}\xspace}
\newcommand{\SwitchBuildLevel}{{\tt SYNEIGHT\_\_SWITCH\_\_BUILD\_LEVEL}\xspace}
\newcommand{\SwitchInline}{{\tt SYNEIGHT\_\_SWITCH\_\_INLINE}\xspace}
\newcommand{\BuildLevel}{{\tt SYNEIGHT\_\_BUILD\_LEVEL}\xspace}
\newcommand{\BuildLevelType}{{\tt Build\_Level\_t}\xspace}
\newcommand{\VerbosityLevelType}{{\tt Verbosity\_Level\_t}\xspace}

\newcommand{\InternalParamBaseFile}{{\tt SYNEIGHT\_\_INTERNAL\_\_PARAM\_\_BASE\_FILE}\xspace}
\newcommand{\InternalParamPrettyFunction}{{\tt SYNEIGHT\_\_INTERNAL\_\_PARAM\_\_PRETTY\_FUNCTION}\xspace}

\newcommand{\InternalParamBaseExceptionType}{{\tt SYNEIGHT\_\_INTERNAL\_\_PARAM\_\_BASE\_EXCEPTION\_TYPE}\xspace}
\newcommand{\InternalParamTestExceptionType}{{\tt SYNEIGHT\_\_INTERNAL\_\_PARAM\_\_TEST\_EXCEPTION\_TYPE}\xspace}
\newcommand{\InternalParamExceptionStrWhat}{{\tt SYNEIGHT\_\_INTERNAL\_\_PARAM\_\_EXCEPTION\_WHAT}\xspace}
\newcommand{\InternalParamTestExceptionStrWhat}{{\tt SYNEIGHT\_\_INTERNAL\_\_PARAM\_\_TEST\_\_EXCEPTION\_WHAT}\xspace}
\newcommand{\InternalSwitchHandleTestExceptionExplicitly}{{\tt SYNEIGHT\_\_INTERNAL\_\_SWITCH\_\_HANDLE\_TEST\_EXCEPTION\_EXPLICITLY}\xspace}

\newcommand{\InternalParamExceptionA}{{\tt SYNEIGHT\_\_INTERNAL\_\_PARAM\_\_EXCEPTION0}\xspace}
\newcommand{\InternalParamExceptionB}{{\tt SYNEIGHT\_\_INTERNAL\_\_PARAM\_\_EXCEPTION1}\xspace}
\newcommand{\InternalParamExceptionC}{{\tt SYNEIGHT\_\_INTERNAL\_\_PARAM\_\_EXCEPTION2}\xspace}
\newcommand{\InternalParamExceptionD}{{\tt SYNEIGHT\_\_INTERNAL\_\_PARAM\_\_EXCEPTION3}\xspace}
\newcommand{\InternalParamExceptionE}{{\tt SYNEIGHT\_\_INTERNAL\_\_PARAM\_\_EXCEPTION4}\xspace}

\newcommand{\InternalParamThrowActionA}{{\tt SYNEIGHT\_\_INTERNAL\_\_PARAM\_\_THROW\_\_ACTION0}\xspace}
\newcommand{\InternalParamThrowActionB}{{\tt SYNEIGHT\_\_INTERNAL\_\_PARAM\_\_THROW\_\_ACTION1}\xspace}
\newcommand{\InternalParamThrowActionC}{{\tt SYNEIGHT\_\_INTERNAL\_\_PARAM\_\_THROW\_\_ACTION2}\xspace}
\newcommand{\InternalParamThrowActionD}{{\tt SYNEIGHT\_\_INTERNAL\_\_PARAM\_\_THROW\_\_ACTION3}\xspace}
\newcommand{\InternalParamThrowActionE}{{\tt SYNEIGHT\_\_INTERNAL\_\_PARAM\_\_THROW\_\_ACTION4}\xspace}

\newcommand{\InternalParamLogActionA}{{\tt SYNEIGHT\_\_INTERNAL\_\_PARAM\_\_LOG\_\_ACTION0}\xspace}
\newcommand{\InternalParamLogActionB}{{\tt SYNEIGHT\_\_INTERNAL\_\_PARAM\_\_LOG\_\_ACTION1}\xspace}
\newcommand{\InternalParamLogActionC}{{\tt SYNEIGHT\_\_INTERNAL\_\_PARAM\_\_LOG\_\_ACTION2}\xspace}
\newcommand{\InternalParamLogActionD}{{\tt SYNEIGHT\_\_INTERNAL\_\_PARAM\_\_LOG\_\_ACTION3}\xspace}
\newcommand{\InternalParamLogActionE}{{\tt SYNEIGHT\_\_INTERNAL\_\_PARAM\_\_LOG\_\_ACTION4}\xspace}


\newcommand{\XXXTrace}{{\tt SYNEIGHT\_\_XXX\_\_TRACE}\xspace}
\newcommand{\XXXBinaryTrace}{{\tt SYNEIGHT\_\_XXX\_\_BINARY\_TRACE}\xspace}

\newcommand{\XXXBlockEnter}{{\tt SYNEIGHT\_\_XXX\_\_BLOCK\_ENTER}\xspace}
\newcommand{\XXXBlockExit}{{\tt SYNEIGHT\_\_XXX\_\_BLOCK\_EXIT}\xspace}
\newcommand{\XXXBlockGuard}{{\tt SYNEIGHT\_\_XXX\_\_BLOCK\_GUARD}\xspace}

\newcommand{\XXXProcedureEnter}{{\tt SYNEIGHT\_\_XXX\_\_PROCEDURE\_ENTER}\xspace}
\newcommand{\XXXProcedureExit}{{\tt SYNEIGHT\_\_XXX\_\_PROCEDURE\_EXIT}\xspace}
\newcommand{\XXXProcedureGuard}{{\tt SYNEIGHT\_\_XXX\_\_PROCEDURE\_GUARD}\xspace}

\newcommand{\XXXMethodEnter}{{\tt SYNEIGHT\_\_XXX\_\_METHOD\_ENTER}\xspace}
\newcommand{\XXXMethodExit}{{\tt SYNEIGHT\_\_XXX\_\_METHOD\_EXIT}\xspace}
\newcommand{\XXXMethodGuard}{{\tt SYNEIGHT\_\_XXX\_\_METHOD\_GUARD}\xspace}



\newcommand{\Inline}{{\tt SYNEIGHT\_\_INLINE}\xspace}

%%% Local Variables: 
%%% mode: latex
%%% TeX-master: t
%%% End: 
% vim:ts=4:sw=4


% vim:ts=4:sw=4

% 
% \begin{document}
% 
% \setTitle{Sample Meeting}
% 
% % set further title page information (omitted in this example)
% 
% \meetingtitlepage
% 
% \agenda
% 
% \topic{Topic 1}{10}
% 
% % topic content
% 
% \topic{Topic 2}{15}
% 
% % topic content
% 
% \protocol
% 
% \topic{Topic 1}{12}
% 
% % topic content
% 
% \topic{Topic 2}{14}
% 
% % topic content
% 
% \end{document}
% 
% ----------------------------------------
% 
% I think this is a little easier to use than the previous
% environment, since the document structure is now really simple.
% 
% Basically, I dropped all the "paperversion" stuff, thus removing the
% agenda/protocol summary. If anybody found those really useful, I'm
% sorry, but in my opinion, these are not really necessary. The agenda
% should not be more than one or two pages long anyway, so there is no
% real need for an extra summary, and the protocol does not need a
% summary at all.
% 
% The main advantage of the new environment is, that content within
% "topics" (formerly "issues", but the command \issue seemingly already
% exists somewhere in latex) can be any valid latex-code (e.g. a
% "verbatim" environment). This is because the content is no longer
% passed as an argument to the command. The structural commands are now
% more like "section"/"subsection" commands.
% 
% In contrast to the old environment, the content of topics is no longer
% restricted to a list of 'subissues', but can be anything the author
% wants to put there.
% 
% Topic headers are still numbered and contain time tracking information
% (start, stop and duration in minutes).
% 
% The title page now contains only a table with general information
% about the meeting (title, date, time, participants, ...). This
% information is set via \setXXX commands (see below) and then displayed
% with the \meetingtitlepage command.
% 
% I also added a command and environment for making notes: \note and
% 'longnote'-environment. These should be used for information that was
% not actually discussed during the meeting, but was added later to the
% protocol.
%
%%%%%%%%%%%%%%%%%%%%%%%%%%%%%%%%%%%%%%%%%%%%%%%%%%%%%%%%%%%%%%%%%%%%%%%%%%%%%%

\documentclass[a4paper, 10pt]{article}

\setcounter{secnumdepth}{3}

\usepackage{fullpage}
\usepackage{xspace}
\usepackage{varioref}
\usepackage{epic}
\usepackage{ecltree}
\usepackage{amsmath}
\usepackage{tabularx}
\usepackage{epsfig}
\usepackage{hhline}
\usepackage{multirow}
\usepackage{hhline}
\usepackage{calc}
\usepackage{ifthen}
\usepackage{graphics}

\sloppy

%%%%%%%%%%%%%%%%%%%%%%%%%%%%%%%%%%%%%%%%%%%%%%%%%%%%%%%%%%%%%%%%%%%%%%%%%%%%%%
%
% Counters used for topic numbering and time tracking.
%
%%%%%%%%%%%%%%%%%%%%%%%%%%%%%%%%%%%%%%%%%%%%%%%%%%%%%%%%%%%%%%%%%%%%%%%%%%%%%%

\newcounter{counttime}
\newcounter{counttopics}

%%%%%%%%%%%%%%%%%%%%%%%%%%%%%%%%%%%%%%%%%%%%%%%%%%%%%%%%%%%%%%%%%%%%%%%%%%%%%%
%
% Commands for setting title-page information
%
% These must be called before \meetingtitlepage is called!
%
%%%%%%%%%%%%%%%%%%%%%%%%%%%%%%%%%%%%%%%%%%%%%%%%%%%%%%%%%%%%%%%%%%%%%%%%%%%%%%

\newcommand{\setTitle}[1]{\def\meetingTitle{#1}}
\newcommand{\setCallers}[1]{\def\meetingCallers{#1}}
\newcommand{\setProposedDate}[1]{\def\meetingProposedDate{#1}}
\newcommand{\setProposedTime}[1]{\def\meetingProposedTime{#1}}
\newcommand{\setProposedAttendants}[1]{\def\meetingProposedAttendants{#1}}
\newcommand{\setProposedModerator}[1]{\def\meetingProposedModerator{#1}}
\newcommand{\setProposedProtocolWriter}[1]{\def\meetingProposedProtocolWriter{#1}}
\newcommand{\setDate}[1]{\def\meetingDate{#1}}
\newcommand{\setTime}[1]{\def\meetingTime{#1}}
\newcommand{\setAttendants}[1]{\def\meetingAttendants{#1}}
\newcommand{\setModerator}[1]{\def\meetingModerator{#1}}
\newcommand{\setProtocolWriter}[1]{\def\meetingProtocolWriter{#1}}

%%%%%%%%%%%%%%%%%%%%%%%%%%%%%%%%%%%%%%%%%%%%%%%%%%%%%%%%%%%%%%%%%%%%%%%%%%%%%%
%
% Command \heading{title} ... internal
%
% This is used to generate a section heading (on the titlepage and at the
% beginnings of the agenda and protocol).
%
%%%%%%%%%%%%%%%%%%%%%%%%%%%%%%%%%%%%%%%%%%%%%%%%%%%%%%%%%%%%%%%%%%%%%%%%%%%%%%

\newcommand{\heading}[1]{%
  \begin{center}%
  {\Large \bf #1}%
  \end{center}%
}

%%%%%%%%%%%%%%%%%%%%%%%%%%%%%%%%%%%%%%%%%%%%%%%%%%%%%%%%%%%%%%%%%%%%%%%%%%%%%%
%
% Command \meetingtitlepage
%
% Generates the title-page, which contains the information set with the
% commands above.
%
%%%%%%%%%%%%%%%%%%%%%%%%%%%%%%%%%%%%%%%%%%%%%%%%%%%%%%%%%%%%%%%%%%%%%%%%%%%%%%

\newcommand{\meetingtitlepage}{%
  \heading{Meeting: \meetingTitle}%
  \bigskip%
  \begin{tabular}{|p{30mm}|p{120mm}|} \hline%
    Callers         & \meetingCallers                \\ \hline%
  \end{tabular}%
  \bigskip\\%
  {\bf Proposed:}%
  \smallskip\\%
  \begin{tabular}{|p{30mm}|p{120mm}|} \hline%
    Date            & \meetingProposedDate           \\ \hline%
    Time            & \meetingProposedTime           \\ \hline%
    Participants    & \meetingProposedAttendants     \\ \hline%
    Moderator       & \meetingProposedModerator      \\ \hline%
    Protocol Writer & \meetingProposedProtocolWriter \\ \hline%
  \end{tabular}%
  \bigskip\\%
  {\bf Actual:}%
  \smallskip\\%
  \begin{tabular}{|p{30mm}|p{120mm}|} \hline%
    Date            & \meetingDate                   \\ \hline%
    Time            & \meetingTime                   \\ \hline%
    Participants    & \meetingAttendants             \\ \hline%
    Moderator       & \meetingModerator              \\ \hline%
    Protocol Writer & \meetingProtocolWriter         \\ \hline%
  \end{tabular}%
}

%%%%%%%%%%%%%%%%%%%%%%%%%%%%%%%%%%%%%%%%%%%%%%%%%%%%%%%%%%%%%%%%%%%%%%%%%%%%%%
%
% Command \newsection{title} ... internal
%
% Used to start a new section (agenda or protocol).
%
%%%%%%%%%%%%%%%%%%%%%%%%%%%%%%%%%%%%%%%%%%%%%%%%%%%%%%%%%%%%%%%%%%%%%%%%%%%%%%

\newcommand{\newsection}[1]{%
  \newpage%
  \setcounter{counttime}{0}%
  \setcounter{counttopics}{1}%
  \heading{#1}%
}

%%%%%%%%%%%%%%%%%%%%%%%%%%%%%%%%%%%%%%%%%%%%%%%%%%%%%%%%%%%%%%%%%%%%%%%%%%%%%%
%
% Command \agenda
%
% Starts the 'Agenda' section.
%
% All topic counters are reset.
%
%%%%%%%%%%%%%%%%%%%%%%%%%%%%%%%%%%%%%%%%%%%%%%%%%%%%%%%%%%%%%%%%%%%%%%%%%%%%%%

\newcommand{\agenda}{%
  \newsection{Agenda}%
}

%%%%%%%%%%%%%%%%%%%%%%%%%%%%%%%%%%%%%%%%%%%%%%%%%%%%%%%%%%%%%%%%%%%%%%%%%%%%%%
%
% Command \protocol
%
% Starts the 'Protocol' section.
%
% All topic counters are reset.
%
%%%%%%%%%%%%%%%%%%%%%%%%%%%%%%%%%%%%%%%%%%%%%%%%%%%%%%%%%%%%%%%%%%%%%%%%%%%%%%

\newcommand{\protocol}{%
  \newsection{Protocol}%
}

%%%%%%%%%%%%%%%%%%%%%%%%%%%%%%%%%%%%%%%%%%%%%%%%%%%%%%%%%%%%%%%%%%%%%%%%%%%%%%
%
% Command \topic{title}{duration}
%
% Starts a new topic.
%
% Arguments:
%
%   title    ... the topic title
%   duration ... the duration in minutes (projected duration for agenda, 
%                actual duration for protocol).
%
%%%%%%%%%%%%%%%%%%%%%%%%%%%%%%%%%%%%%%%%%%%%%%%%%%%%%%%%%%%%%%%%%%%%%%%%%%%%%%

\newcommand{\topic}[2]{%
  \bigskip%
  \noindent%
  \begin{tabular}{|p{7mm}|p{110mm}|p{35mm}|}%
  \hline%
  \hfill\thecounttopics & \textbf{#1} & \hfill #2 min (\thecounttime\addtocounter{counttime}{#2}--\thecounttime)%
  \\%
  \hline%
  \end{tabular}%
  \medskip%
  \addtocounter{counttopics}{1}%
}

%%%%%%%%%%%%%%%%%%%%%%%%%%%%%%%%%%%%%%%%%%%%%%%%%%%%%%%%%%%%%%%%%%%%%%%%%%%%%%
%
% Note commands:
%
%   \note{user}{text}
%
%   \begin{longnote}{user}
%      % text
%   \end{longnote}
%
% These should be used to add information to the protocol that was _not_
% discussed during the meeting (e.g. additional information or notions added
% by the protocol writer or a reviewer).
%
% Use \note for short (one sentence) notes and longnote-environment for
% longer notes (this creates a quote-environment around the note's content).
%
% Arguments:
%
%   user ... the name of the user that added the note.
%   text ... the note's content.
%
%%%%%%%%%%%%%%%%%%%%%%%%%%%%%%%%%%%%%%%%%%%%%%%%%%%%%%%%%%%%%%%%%%%%%%%%%%%%%%

\newcommand{\note}[2]{%
  [{\bf Note by #1:} #2]%
}

\newenvironment{longnote}[1]{%
  {\bf Note by #1:}%
  \begin{quote}%
}{%
  \end{quote}%
}

%%%%%%%%%%%%%%%%%%%%%%%%%%%%%%%%%%%%%%%%%%%%%%%%%%%%%%%%%%%%%%%%%%%%%%%%%%%%%%
%
% Include standard files
%
%%%%%%%%%%%%%%%%%%%%%%%%%%%%%%%%%%%%%%%%%%%%%%%%%%%%%%%%%%%%%%%%%%%%%%%%%%%%%%

%------------------------------------------------------------------------------
% Syneight - A soft-realtime transaction monitor.
% Copyright (C) 2003-2004 The Syneight Group.
%
% TODO: License.
%------------------------------------------------------------------------------

\newcommand{\CS}{\textsc{Christian}\xspace}
\newcommand{\PET}{\textsc{Piotr}\xspace}
\newcommand{\OLLI}{\textsc{Olli}\xspace}
\newcommand{\TOBI}{\textsc{Tobi}\xspace}
\newcommand{\UWE}{\textsc{Uwe}\xspace}
\newcommand{\DHR}{\textsc{Daniel}\xspace}
\newcommand{\ANDREAS}{\textsc{Andreas}\xspace}

\newcommand{\TEAM}{\CS \PET \OLLI \TOBI \UWE \DHR \ANDREAS}

%%% Local Variables: 
%%% mode: latex
%%% TeX-master: t
%%% End: 
% vim:ts=4:sw=4

%------------------------------------------------------------------------------
% Syneight - A soft-realtime transaction monitor.
% Copyright (C) 2003-2004 The Syneight Group.
%
% TODO: License.
%------------------------------------------------------------------------------

\usepackage{xspace}

\newcommand{\SYNEIGHT}{\textsc{Syneight}\xspace}
\newcommand{\DSM}{\textsc{Dsm}\xspace}

\newcommand{\MMOG}{\textsc{Mmog}\xspace}
\newcommand{\MMOGS}{\textsc{Mmog}s\xspace}
\newcommand{\MMORG}{\textsc{Mmorg}\xspace}
\newcommand{\MMORGS}{\textsc{Mmorg}s\xspace}
\newcommand{\NVE}{\textsc{Nve}\xspace}
\newcommand{\NVES}{\textsc{Nve}s\xspace}

\newcommand{\DBAI}{{ \normalfont\textsc{dbai}}\xspace}

\newcommand{\EmeLevel}{{\tt EME\_LEVEL}\xspace}
\newcommand{\AleLevel}{{\tt ALE\_LEVEL}\xspace}
\newcommand{\ProLevel}{{\tt PRO\_LEVEL}\xspace}
\newcommand{\DebLevel}{{\tt DEB\_LEVEL}\xspace}
\newcommand{\AudLevel}{{\tt AUD\_LEVEL}\xspace}
\newcommand{\TesLevel}{{\tt TES\_LEVEL}\xspace}
\newcommand{\SysLevel}{{\tt SYS\_LEVEL}\xspace}

\newcommand{\SwitchTestAnnotations}{{\tt SYNEIGHT\_\_SWITCH\_\_TEST\_ANNOTATIONS}\xspace}
\newcommand{\SwitchProductionAnnotations}{{\tt SYNEIGHT\_\_SWITCH\_\_PRODUCTION\_ANNOTATIONS}\xspace}
\newcommand{\SwitchLogProChecks}{{\tt SYNEIGHT\_\_SWITCH\_\_LOG\_PRO\_CHECKS}\xspace}
\newcommand{\SwitchBuildLevel}{{\tt SYNEIGHT\_\_SWITCH\_\_BUILD\_LEVEL}\xspace}
\newcommand{\SwitchInline}{{\tt SYNEIGHT\_\_SWITCH\_\_INLINE}\xspace}
\newcommand{\BuildLevel}{{\tt SYNEIGHT\_\_BUILD\_LEVEL}\xspace}
\newcommand{\BuildLevelType}{{\tt Build\_Level\_t}\xspace}
\newcommand{\VerbosityLevelType}{{\tt Verbosity\_Level\_t}\xspace}

\newcommand{\InternalParamBaseFile}{{\tt SYNEIGHT\_\_INTERNAL\_\_PARAM\_\_BASE\_FILE}\xspace}
\newcommand{\InternalParamPrettyFunction}{{\tt SYNEIGHT\_\_INTERNAL\_\_PARAM\_\_PRETTY\_FUNCTION}\xspace}

\newcommand{\InternalParamBaseExceptionType}{{\tt SYNEIGHT\_\_INTERNAL\_\_PARAM\_\_BASE\_EXCEPTION\_TYPE}\xspace}
\newcommand{\InternalParamTestExceptionType}{{\tt SYNEIGHT\_\_INTERNAL\_\_PARAM\_\_TEST\_EXCEPTION\_TYPE}\xspace}
\newcommand{\InternalParamExceptionStrWhat}{{\tt SYNEIGHT\_\_INTERNAL\_\_PARAM\_\_EXCEPTION\_WHAT}\xspace}
\newcommand{\InternalParamTestExceptionStrWhat}{{\tt SYNEIGHT\_\_INTERNAL\_\_PARAM\_\_TEST\_\_EXCEPTION\_WHAT}\xspace}
\newcommand{\InternalSwitchHandleTestExceptionExplicitly}{{\tt SYNEIGHT\_\_INTERNAL\_\_SWITCH\_\_HANDLE\_TEST\_EXCEPTION\_EXPLICITLY}\xspace}

\newcommand{\InternalParamExceptionA}{{\tt SYNEIGHT\_\_INTERNAL\_\_PARAM\_\_EXCEPTION0}\xspace}
\newcommand{\InternalParamExceptionB}{{\tt SYNEIGHT\_\_INTERNAL\_\_PARAM\_\_EXCEPTION1}\xspace}
\newcommand{\InternalParamExceptionC}{{\tt SYNEIGHT\_\_INTERNAL\_\_PARAM\_\_EXCEPTION2}\xspace}
\newcommand{\InternalParamExceptionD}{{\tt SYNEIGHT\_\_INTERNAL\_\_PARAM\_\_EXCEPTION3}\xspace}
\newcommand{\InternalParamExceptionE}{{\tt SYNEIGHT\_\_INTERNAL\_\_PARAM\_\_EXCEPTION4}\xspace}

\newcommand{\InternalParamThrowActionA}{{\tt SYNEIGHT\_\_INTERNAL\_\_PARAM\_\_THROW\_\_ACTION0}\xspace}
\newcommand{\InternalParamThrowActionB}{{\tt SYNEIGHT\_\_INTERNAL\_\_PARAM\_\_THROW\_\_ACTION1}\xspace}
\newcommand{\InternalParamThrowActionC}{{\tt SYNEIGHT\_\_INTERNAL\_\_PARAM\_\_THROW\_\_ACTION2}\xspace}
\newcommand{\InternalParamThrowActionD}{{\tt SYNEIGHT\_\_INTERNAL\_\_PARAM\_\_THROW\_\_ACTION3}\xspace}
\newcommand{\InternalParamThrowActionE}{{\tt SYNEIGHT\_\_INTERNAL\_\_PARAM\_\_THROW\_\_ACTION4}\xspace}

\newcommand{\InternalParamLogActionA}{{\tt SYNEIGHT\_\_INTERNAL\_\_PARAM\_\_LOG\_\_ACTION0}\xspace}
\newcommand{\InternalParamLogActionB}{{\tt SYNEIGHT\_\_INTERNAL\_\_PARAM\_\_LOG\_\_ACTION1}\xspace}
\newcommand{\InternalParamLogActionC}{{\tt SYNEIGHT\_\_INTERNAL\_\_PARAM\_\_LOG\_\_ACTION2}\xspace}
\newcommand{\InternalParamLogActionD}{{\tt SYNEIGHT\_\_INTERNAL\_\_PARAM\_\_LOG\_\_ACTION3}\xspace}
\newcommand{\InternalParamLogActionE}{{\tt SYNEIGHT\_\_INTERNAL\_\_PARAM\_\_LOG\_\_ACTION4}\xspace}


\newcommand{\XXXTrace}{{\tt SYNEIGHT\_\_XXX\_\_TRACE}\xspace}
\newcommand{\XXXBinaryTrace}{{\tt SYNEIGHT\_\_XXX\_\_BINARY\_TRACE}\xspace}

\newcommand{\XXXBlockEnter}{{\tt SYNEIGHT\_\_XXX\_\_BLOCK\_ENTER}\xspace}
\newcommand{\XXXBlockExit}{{\tt SYNEIGHT\_\_XXX\_\_BLOCK\_EXIT}\xspace}
\newcommand{\XXXBlockGuard}{{\tt SYNEIGHT\_\_XXX\_\_BLOCK\_GUARD}\xspace}

\newcommand{\XXXProcedureEnter}{{\tt SYNEIGHT\_\_XXX\_\_PROCEDURE\_ENTER}\xspace}
\newcommand{\XXXProcedureExit}{{\tt SYNEIGHT\_\_XXX\_\_PROCEDURE\_EXIT}\xspace}
\newcommand{\XXXProcedureGuard}{{\tt SYNEIGHT\_\_XXX\_\_PROCEDURE\_GUARD}\xspace}

\newcommand{\XXXMethodEnter}{{\tt SYNEIGHT\_\_XXX\_\_METHOD\_ENTER}\xspace}
\newcommand{\XXXMethodExit}{{\tt SYNEIGHT\_\_XXX\_\_METHOD\_EXIT}\xspace}
\newcommand{\XXXMethodGuard}{{\tt SYNEIGHT\_\_XXX\_\_METHOD\_GUARD}\xspace}



\newcommand{\Inline}{{\tt SYNEIGHT\_\_INLINE}\xspace}

%%% Local Variables: 
%%% mode: latex
%%% TeX-master: t
%%% End: 
% vim:ts=4:sw=4


% vim:ts=4:sw=4
