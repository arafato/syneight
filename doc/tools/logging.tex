%------------------------------------------------------------------------------
% Syneight - A soft-realtime transaction monitor.
% Copyright (C) 2003-2004 The Syneight Group.
%
% TODO: License.
%------------------------------------------------------------------------------

\chapter{Log4Cpp - A short overview}

\section{Appenders}
All Appender objects are inherited from {\em Appender}. {\em LayoutAppender} is the common superclass
for all Appenders that require a Layout. Log4Cpp allows a Category object to log messages to multiple destinations
by using an appender object. It provides Appenders for:
\begin{itemize}
	\item{Files}
	\item{OStreams}
	\item{UNIX syslog daemons}
	\item{Remote UNIX syslog daemons}
\end{itemize}

\subsection {FileAppender}
A FileAppender allows you to log to a file.
\begin{verbatim}
 log4cpp::FileAppender::FileAppender(const std::string & name,
		                     const std::string & fileName,
		                     bool append = true,
		                     mode_t mode = 00644
                                    )
\end{verbatim}  	

\begin{description}
	\item[name] the name of the Appender.
 	\item[fileName]	the name of the file to which the Appender has to log.
	\item[append] 	whether the Appender has to truncate the file or just append to it if it already exists. 
			Defaults to 'true'.
	\item[mode]	file mode to open the logfile with. Defaults to 00644. 
\end{description}

\subsection{OstreamAppender}
A OstreamAppender allows you to log to any ostream object. 
\begin{verbatim}
log4cpp::OstreamAppender::OstreamAppender(const std::string & name,
					  std::ostream * stream
					 )  	
\end{verbatim}
\begin{description}
	\item[name] the name of the Appender.
 	\item[stream] the stream object to log to.
\end{description}


\subsection{SyslogAppender}
A SyslogAppender allows you to log to a local syslog daemon. 
Note that the C syslog API is process global, so instantion of a 
second SyslogAppender will overwrite the syslog name of the first.
\begin{verbatim}
log4cpp::SyslogAppender::SyslogAppender(const std::string & name,
		const std::string & syslogName,
		int facility = LOG_USER
		)  
\end{verbatim}
\begin{description}
	\item[name] the name of the Appender.
 	\item[syslogName] the ident parameter in the openlog(3) call.
	\item[facility] the syslog facility to log to. Defaults to LOGUSER
\end{description}

\subsection{SyslogAppender}
RemoteSyslogAppender sends LoggingEvents to a remote syslog system.
\begin{verbatim}
log4cpp::RemoteSyslogAppender::RemoteSyslogAppender(const std::string & name,
		const std::string & syslogName,
		const std::string & relayer,
		int facility = LOG_USER,
		int portNumber = 514
	)  
\end{verbatim}
\begin{description}
	\item[name] the name of the Appender.
 	\item[syslogName] the ident parameter in the openlog(3) call.
	\item[relayer] the IP address or hostname of a standard syslog host
	\item[facility] the syslog facility to log on. Defaults to LOGUSER. Value -1 implies to
			use the default.
	\item[portNumber] An alternative port number. Defaults to the standard syslog port number(514).
			  Value -1 implies to use the default.
\end{description}


\section{Layout}
A Layout object tells an Appender object how to format the log. Currently there are three different Layout
classes available which are all inherited from the abstract base class {\em Layout}. Please note that every single Appender
expects his own layout object! Otherwise you will get a {\em Segmentation fault} and logging won't work properly!
\begin{itemize}	
	\item{SimpleLayout}
	\item{BasicLayout}	
	\item{PatternLayout}
\end{itemize}

LoggingEvents are log4cpp's internal representations of a log message! See Section 4 for more
information!

\subsection{SimpleLayout}
SimpleLayout formats the LoggingEvent in the following style:\\
{\sc priority - message}\\
{\bf Usage:}
\begin{verbatim}
log4cpp::Layout * simpleLayout = new log4cpp::SimpleLayout();
\end{verbatim}

\subsection{BasicLayout}
BasicLayout formats the LoggingEvent in the following style:\\
{\sc timeStamp - priority - category - ndc: message}\\
{\bf Usage:}
\begin{verbatim}
log4cpp::Layout * basicLayout = new log4cpp::BasicLayout();
\end{verbatim}

\subsection{PatternLayout}
PatternLayout gives you a lot of control on how the LoggingEvent is to look like.
You can choose from a huge variety of options.\\
{\bf Usage:}\\
First of all you have to initiate a PatternLayout object.
\begin{verbatim}
log4cpp::Layout* patternLayout = new log4cpp::PatternLayout();
\end{verbatim}
To set the format of the log lines handled by this PatternLayout, you have to call
\begin{verbatim}
void log4cpp::PatternLayout::
		setConversionPattern(const std::string & conversionPattern) 
	        throw (configureFailure)
\end{verbatim}
Please note that the method {\em setConversionPattern} is not inherited from the base class {\em Layout}!\\
A {\em configureFailure} exception is thrown if the conversionPattern is invalid.
{\bf conversionPattern} can be composed from the following characters:
\begin{itemize}
	\item{\%c - the category}
	\item{\%d - the date}
	\item{\%m - the message}
        \item{\%n - the platform specific line separator}
        \item{\%p - the priority}
        \item{\%r - milliseconds since this layout was created}
       	\item{\%R - seconds since Jan 1, 1970}
       	\item{\%u - clock ticks since process start}
       	\item{\%x - the NDC}
\end{itemize}
{\bf Date format:}\\
The date format character may be followed by a date format specifier enclosed between braces. 
For example, \%d{\%H:\%M:\%S,\%l} or \%d{\%d \%m \%Y \%H:\%M:\%S,\%l}. If no date format specifier is given then the following
format is used: "Wed Jan 02 02:03:55 1980". The date format specifier admits the same syntax as the ANSI C
function strftime, with 1 addition. The addition is the specifier \%l for milliseconds, padded with zeros to make 3 digits.

\section{Categories and Priorities}
The base class {\em Category} provides many methods to perform the actual logging.
You have to assign a priority to it. Furthermore every log message itself has a priority.
Depending on that, log messages are skipped or processed by the category object if the message
priority is lower or equal than the category priority. When a category object is created, it starts life with a 
default appender of standard out and a default priority of {\em Notset}. \\
The enum {\em PriorityLevel} is defined in the base class {\em Priority}. The following levels are defined:
\begin{description}
	\item{{\sc Emerg = 0}}
	\item{{\sc Fatal = 0}}
	\item{{\sc Alert = 100}}
	\item{{\sc Crit = 200}}
	\item{{\sc Error = 300}}
	\item{{\sc Warn = 400}}
	\item{{\sc Notice = 500}}
	\item{{\sc Info = 600}}
	\item{{\sc Debug = 700}}
	\item{{\sc Notset = 800}}
\end{description}
By default the priority is set to {\sc Notset}.\\
The constructor of {\em Category} is protected, so you have to to use the following static method to instantiate a
child category object.
The Category class creates one overall static root category object. Please note that the root category does not have a name!
The application can create instances of children categories from this one static parent.
\begin{verbatim}
Category & log4cpp::Category::getInstance(const std::string & name)
\end{verbatim}
\begin{description}
	\item[name] the name of the category to retrieve
\end{description}
If you want to create a child {\sc sub2} from a parent {\sc sub1}, you have to write:
\begin{verbatim}
log4cpp::Category & sub2 = log4cpp::Category::getInstance(``SUB1.SUB2'');
\end{verbatim}
By default, root is always the parent!
Note that this method does not set the priority of category!\\
Use the method {\em Category::setPriority} to set the Priority Level:
\begin{verbatim}
void log4cpp::Category::setPriority(Priority::Value priority) 
				    throw (std::invalid_argument) 
\end{verbatim}
\begin{description}
	\item[priority] the Priority to set. Use {\em Priority::NOTSET} to let the category use its parents
			priority as effective priority.
	\item[Exception: std::invalid\_argument] if the caller tries to set {\em Priority::NOTSET} on the
						Root Category.
\end{description}
In contrast to {\sc log4j} where the priority of the parent is always inherited to the child if not stated otherwise, 
log4cpp does not offer this feature! Therefore you have to set the priority for every category object manually!

\subsection{How to generate log messages}
There are several ways to let log4cpp generate log messages. There is the simple log()-method,
mnemonic methods for the severity level and stream operations.

\subsubsection{The log()-method}
\begin{verbatim}
void log4cpp::Category::log(Priority::Value priority,
		const std::string & message
		)throw ()
\end{verbatim}
\begin{description}
	\item[priority] the priority of this message
	\item[message] the log message
\end{description}

\subsubsection{Mnemonic methods}
The name of the method itself indicates the severity level.
The name is equal to the priority level. As every log message has to have
a priority assigned to it, there does not exist a method called {\em notset}, of course.
The log message is simply passed by an argument, i.e if you want to generate a message
with priority {\sc Alert} the syntax is:
\begin{verbatim}
void log4cpp::Category::alert(const std::string & message) throw ()	
\end{verbatim}

\subsubsection{Stream style operations}
There are two possibilities how to use stream operations.
They can be used directly with the category object and are preceded by the severity level.
For instance, suppose you want to generate a log message with priority {\em error} assigned to it.
\begin{verbatim}
cat_obj << log4cpp::Priority::ERROR << "logmessage" << 
	log4cpp::CategoryStream::ENDLINE;
\end{verbatim}
Note that {\em CategoryStream::ENDLINE} is part of the enumeration {\em Separators}. 
Currently it only contains the 'ENDLINE' separator, which separates two log messages.

The second possibility is to use a method that will return a {\em CategoryStream} with its priority
already assigned to it. Priority depends - once again - on the method you use, so for every priority level there
exists a method(<Prioritylevel>Stream()). Suppose you want to generate a log message with priority {\em debug}.
\begin{verbatim}
cat_obj.debugStream() << "logmessage" << log4cpp::CategoryStream::ENDLINE;
\end{verbatim}

\section{LoggingEvents}
As already mentioned earlier above a logging event is log4cpps internal representation of a log message.
The base class {\em LoggingEvent} only consists of one public constructor(has no destructor!).
\begin{verbatim}
log4cpp::LoggingEvent::LoggingEvent(const std::string & category,
		const std::string & message,
		const std::string & ndc,
		Priority::Value priority
		)  	
\end{verbatim} 
\begin{description}
	\item[category] The category of this event.
	\item[message] The message of this event.
	\item[ndc] The nested diagnostic context.
	\item[priority] The priority of this event.\\
\end{description}
Furthermore, {\em LoggingEvent} has the following public attributes:
\begin{description}
	\item[const string categoryName] The category name.
	\item[const string message] The application supplied message of logging event.
	\item[const string ndc] The nested diagnostic context of logging event.
	\item[Priority::PriorityLevel priority] Priority of logging event.
	\item[const string threadName] The name of thread in which this logging event was generated, e.g.
				       the PID. 
	\item[TimeStamp timeStamp] The number of seconds elapsed since the epoch (1/1/1970 00:00:00 UTC) 
				   until logging event was created.
\end{description}

\section{Configuration files}
TODO

\section{Filters}
TODO

\section{How does this all hang together?}
Basically, log4cpp has three main components:
\begin{itemize}
	\item{Appenders}
	\item{Layouts}
	\item{Categories}
\end{itemize}
An {\sc appender class} writes the log message out to some device, which depends on the appender class you've chosen, of course.\\\\
A {\sc layout class} controls what the message is going to look like.\\\\
A {\sc category class} does the actual logging. The two main parts of a category are its appenders and its priority. Priority
controls which messages can be logged by a particular class. One or more appenders can be added to the list of destinations for
logging. An appender when added to a category becomes an additional output destination unless the {\em Additivity Flag} is
set to false. If it is false, the appender added to the category replaces all previously existing appenders. Please note that
every log message of a category child will also be passed to its parent!

Basically there are six initial steps to using a log4cpp log:
\begin{enumerate}
	\item{Instantiate an appender object.}
	\item{Instantiate a layout object.}
	\item{Attach the layout object to the appender (appender->setLayout(layout);}
	\item{Instantiate a category object.}
	\item{Attach the appender object to the category either as an additional appender or as the only appender(depends on
		{\em Additivity Flag})}
	\item{Set a priority for the category unless you simply want to log everything}
\end{enumerate}



%\end{itemize}
%\begin{description}
%\item[Oberpunkt1]$\;$
%	\begin{description}
%		\item[bla:] Erkl�rung 1
%		\item[bla2:] Erkl�rung 2
%	\end{description}
%\end{description}
%\end{itemize}

% vim:ts=4:sw=4
