%------------------------------------------------------------------------------
% Syneight - A soft-realtime transaction monitor.
% Copyright (C) 2003-2004 The Syneight Group.
%
% TODO: License.
%------------------------------------------------------------------------------

\chapter{Subversion}
\label{cha:subversion}

\section{Introduction}
\label{sec:introduction}

\section{Commit Guidelines}
\label{sec:commit-guidelines}

\subsection{When To Commit}
\label{sec:when-to-commit}

\begin{guideline*}{Commit early, commit often}
  Try to commit your changes early and often, so that the diffs
  are small (but cohesive). Do not commit the work of several hours/days at
  once, especially not if the changes you made are unrelated to each other.
\end{guideline*}

\begin{guideline*}{Run {\tt svn update} before you commit}
  Always run {\tt svn update} before committing to make sure the file(s)
  you commit are up to date.
  Otherwise, you risk overwriting changes done by other developers in the
  meantime.
  You possibly will need to merge your changes with those of the other
  developers, if you changed the same lines of a file like the other
  developers.
  The longer it takes before you update, the harder it'll be for SVN
  to merge your changes with the changes done by other developers.
\end{guideline*}

\begin{guideline*}{Check exactly what you commit}
  Always check with {\tt svn diff} what exactly you are committing.
  {\tt svn diff} is also very useful to get an overview of what you changed
  and hence should mention in the commit message. Do \emph{not} just write
  the commit logs from what you remember or think you changed!
  Always use {\tt svn diff}!
\end{guideline*}

\begin{guideline*}{Test your changes before you commit}
  Everything that worked before the commit should still work after the commit.
  E.g. all code should still compile after the commit. Also, if you change
  the documentation, {\tt make docs} should still work.
\end{guideline*}

\begin{guideline*}{One issue per commit}
  Commit one cohesive feature, bugfix or change at a time. Do \emph{not}
  mix a bunch of unrelated changes in the same commit.
  This makes it easier to review changes, and - if necessary - undo specific
  changes.
  If a single, cohesive change involves several files, commit all of them
  in the same commit.
\end{guideline*}


\subsection{Commit Log Messages}
\label{sec:commit-log-messages}

Each commit should have a brief but concise and informative comment.


\section{Branches}
\label{sec:branches}

\section{Tags}
\label{sec:tags}

\section{Code}
\label{sec:code}

\section{ID Strings}
\label{sec:id-strings}


%%% Local Variables: 
%%% mode: latex
%%% TeX-master: "tools"
%%% End: 
% vim:ts=4:sw=4
